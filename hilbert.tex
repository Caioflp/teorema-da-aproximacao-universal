% !TeX root = ./main.tex
\section{O Décimo Terceiro Problema de Hilbert}

Diferentemente do Teorema da Aproximação de Weierstrass, o \( 13^{ \circ } \) problema de Hilbert não trata de aproximações, mas de representações exatas de funções.
Mais especificamente, Hilbert postulou (utilizando a linguagem matemática de sua época) que existem funções contínuas de \( \I^{ 3 } \) em \( \R \), onde \( \I = [0, 1] \), que não podem ser expressas por meio da composição e adição de funções de \( \R^{ 2 } \) em \( \R \).

Décadas após ser postulada, essa conjectura eventualmente foi demonstrada \emph{falsa}.
A prova foi dada por Vladimir Igorevich Arnol'd, 14 anos após a morte de Hilbert.
Ele e seu orientador de Doutorado, Andrej Nikolajewitsch Kolmogorov, provaram que, na verdade, toda função contínua \( f : \I^{ n } \to \R \) pode ser expressa como composições e adições de funções contínuas de \( \R \to \R \).
Com o trabalho de outros matemáticos, esse resultado foi generalizado.
Uma dessas generalizações é apresentada no Teorema a seguir.
\begin{teo}[Kolmogorov, Arnol'd, Kahane, Lorentz e Sprecher]
    Para todo \( n \in \N \) com \( n \geq 2 \), existem números reais \( \lambda_{ 1 }, \lambda_{ 2 }, \dots, \lambda_{ n } \) e funções contínuas \( \varphi_{ k } : \I \to \R \), para \( k = 1, \dots, 2n + 1 \), com a propriedade de que para toda função contínua \( f : \I^{ n } \to \R \) existe uma função contínua \( g : \R \to \R \) tal que, para todo \( ( x_{ 1 }, \dots, x_{ n } ) \in \I^{ n } \),
    \begin{equation}
        f(x_{ 1 }, \dots, x_{ n }) =
        \sum_{ k=1 }^{ 2n+1 } g ( \lambda_{ 1 } \varphi_{ k } ( x_{ 1 } )  + \cdots + \lambda_{ n } \varphi_{ k } ( x_{ n } ) )
        \label{eq: kolmogorov_2}
    .\end{equation}
\end{teo}

Nos atentaremos ao caso especial em que \( n = 2 \):

\begin{teo}
    Existem \( \lambda \in \R \) e funções \( ( \varphi_{ 1 }, \dots, \varphi_{ 5 } ) \in [C(\I)]^{ 5 } \) tais que, para toda função \( f \in C(\I^{ 2 }) \) existe \( g : \R \to \R \), contínua, satisfazendo, para todos \( ( x_{ 1 }, x_{ 2 } ) \in \I^{ 2 } \), \[
        f ( x_{ 1 }, x_{ 2 } ) =
        \sum_{ k=1 }^{ 5 } g ( \varphi_{ k } ( x_{ 1 } ) + \lambda \varphi_{ k } ( x_{ 2 } ) )
    .\]
\end{teo}

Para a prova que apresentaremos, necessitamos de alguns lemas
 