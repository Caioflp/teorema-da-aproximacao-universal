% !TeX root = ./main.tex
\section{Décimo Terceiro Problema de Hilbert}

Aqui provaremos a possibilidade de representar qualquer função contínua \( f : \I^{ n } \to \R \) por meio de composições e adições de funções contínuas de \( \R \to \R \) .
Com o trabalho de vários matemáticos, esse resultado foi generalizado.
Uma dessas generalizações é apresentada no teorema a seguir.

O leitor interassado poderá encontrar mais informações em \cite{hilbert}.
\begin{teo}[Kolmogorov, Arnol'd, Kahane, Lorentz e Sprecher]
    Para todo \( n \in \N \) com \( n \geq 2 \), existem números reais \( \lambda_{ 1 }, \lambda_{ 2 }, \dots, \lambda_{ n } \) e funções contínuas \( \varphi_{ k } : \I \to \R \), para \( k = 1, \dots, 2n + 1 \), com a propriedade de que para toda função contínua \( f : \I^{ n } \to \R \) existe uma função contínua \( g : \R \to \R \) tal que, para todo \( ( x_{ 1 }, \dots, x_{ n } ) \in \I^{ n } \),
    \begin{equation}
        f(x_{ 1 }, \dots, x_{ n }) =
        \sum_{ k=1 }^{ 2n+1 } g ( \lambda_{ 1 } \varphi_{ k } ( x_{ 1 } )  + \cdots + \lambda_{ n } \varphi_{ k } ( x_{ n } ) )
        \label{eq: kolmogorov_2}
    .\end{equation}
\end{teo}
\begin{rem}
    Denotamos o espaço das funções contínuas de \( \I^{ n } \) em \( \R \) por \( C(\I^{ n }) \).
    Notamos que esse espaço, com a norma do supremo, se torna um espaço métrico completo, com a distância dada por
    \begin{equation}
        d(f, g) = \norm{ f - g }_{ \infty }
    .\end{equation}
    O leitor pode encontrar resultados e definições elementares relativas a espaços métricos (inclusive o teorema da categoria de Baire, que será usado a seguir) no apêndice \ref{ap: espacos_metricos}.
\end{rem}
Aqui nos atentaremos ao caso especial em que \( n = 2 \):
\begin{teo}
    Existem \( \lambda \in \R \) e funções \( ( \varphi_{ 1 }, \dots, \varphi_{ 5 } ) \in [C(\I)]^{ 5 } \) tais que, para toda função \( f \in C(\I^{ 2 }) \) existe \( g : \R \to \R \), contínua, satisfazendo, para todos \( ( x_{ 1 }, x_{ 2 } ) \in \I^{ 2 } \), \[
        f ( x_{ 1 }, x_{ 2 } ) =
        \sum_{ k=1 }^{ 5 } g ( \varphi_{ k } ( x_{ 1 } ) + \lambda \varphi_{ k } ( x_{ 2 } ) )
    .\]
    \label{teo: hilbert13}
\end{teo}
\begin{rem}
    Por uma questão de economia de notação, definimos \[
        \Phi_{ k }(x, y) = \sum_{ k=1 }^{ 5 } \varphi_{ k } (x) + \lambda \varphi_{ k } (y)
    .\]
\end{rem}
Para a prova que apresentaremos, necessitamos de alguns lemas.
O primeiro deles clarifica a existência e a escolha de \( \lambda \).
\begin{lem}
    Existe um número real \( \lambda \) tal que, para quaisquer \( x_{ 1 }, x_{ 2 }, x_{ 3 }, x_{ 4 } \in \Q \), \[
        x_{ 1 } + \lambda y_{ 1 } = x_{ 2 } + \lambda y_{ 2 }
        \text{ implica }
        x_{ 1 } = x_{ 2 } \text{ e } y_{ 1 } = y_{ 2 }
    .\]
    \label{lem: lambda}
\end{lem}
\begin{proof}
    Basta escolher \( \lambda \in \R \backslash \Q \), pois se \( x_{ 1 } + \lambda y_{ 1 } = x_{ 2 } + \lambda x_{ 2 } \), então \( x_{ 1 } - x_{ 2 } = \lambda ( y_{ 2 } - y_{ 1 } ) \).
    Como \( x_{ 1 } - x_{ 2 } \in \Q \) necessariametente o lado direito deve ser \( 0 \), ou seja, \( y_{ 2 } = y_{ 1 } \) e \( x_{ 2 } = x_{ 1 } \).
\end{proof}
\begin{lem}
    Fixe \( \lambda \) satisfazendo o lema \ref{lem: lambda}.
    Seja \( f \in C(\I^2) \) com \( \norm{ f }_{ \infty } = 1 \).
    Seja \( U_{ f } \) o suconjunto de \( [C(\I)]^{ 5 } \) tal que \( ( \varphi_{ 1 }, \dots, \varphi_{ 5 } ) \in U_{ f } \) se, e somente se, existe uma \( g \in C(\R) \) tal que
    \begin{equation}
        \abs{ g(t) } \leq \frac{ 1 }{ 7 } \text{ para todo } t \in \R
        \label{eq: g_bound}
    ,\end{equation}
    e
    \begin{equation}
        \abs{ 
            f(x, y)
            - \sum_{ i=1 }^{ 5 } (g \circ \Phi_{ k })(x, y) )
         }
         < \frac{ 7 }{ 8 }, \text{ para todo } x, y \in \I^2
         \label{eq: dif_bound}
    .\end{equation}
    Então \( U_{ f } \) é um suconjunto aberto e denso de \( [C(\I)]^{ 5 } \).
    \label{lem: uf}
\end{lem}
Enquanto é fácil perceber que \( U_{ f } \) é aberto, pois se \( \varphi = ( \varphi_{ 1 }, \dots, \varphi_{ 5 } ) \) satisfaz (\ref{eq: g_bound}) e (\ref{eq: dif_bound}) para uma dada \( g \), então todas as tuplas suficientemente perto de \( \varphi \) também satisfarão, a prova da densidade é mais extensa e será omitida, por não ser do interesse desse trabalho.
Nos restringimos a mencionar que ela necessita do lema \ref{lem: lambda}.
%% Colocar referência de onde ela pode ser encontrada
\begin{lem}
    Seja \( \lambda \) como no lema \ref{lem: lambda}.
    Então existem \( \varphi_{ 1 }, \dots, \varphi_{ 5 } \in C(\I) \) com a propriedade de que, dada \( f \in C(\I^2) \), existe uma \( g \in C(\R) \) satisfazendo \[
        \abs{ g(t) } \leq \frac{ 1 }{ 7 } \norm{ f }_{ \infty } \text{ para todo } t \in \R
    ,\]
    e \[
        \norm{ 
            f - \sum_{ k=1 }^{ 5 } g \circ \Phi_{ k }
         }_{ \infty } < \frac{ 8 }{ 9 } \norm{ f }_{ \infty }
    .\]
    \label{lem: baire}
\end{lem}
\begin{proof}
    Podemos supor, sem perda de generalidade, que \( \norm{ f }_{ \infty } = 1 \).
    Seja \( (h_{ j })_{ j\in\N } \) uma sequência de funções pertencentes a \( C(\I^2) \) tal que o conjunto \( \left\{ h_{ j } : j\in \N \right\} \) é denso na esfera unitária de \( C(\I^2) \).
    Na notação do lema \ref{lem: uf}, cada \( h_{ j } \) determina um conjunto \( U_{ j } = U_{ h_{ j } } \subset  [C(\I)]^{ 5 } \) aberto e denso.
    Pelo teorema da categoria de Baire, o conjunto \[
        V = \bigcap_{ j\in\N } U_{ j }
    \]
    é denso em \( [C(\I)]^{ 5 } \).
    Pela densidade de \( \left\{ h_{ j } : j \in \N \right\} \), existe \( m \in \N \) tal que \( \norm{ f - h_{ m } }_{ \infty } < 1/72 \).
    Além disso, pelo lema \ref{lem: uf} podemos tomar \( ( \varphi_{ 1 }, \dots, \varphi_{ 5 } ) \in V \subset  U_{ m } \) tal que exista \( g \in C(\I) \) satisfazendo \[
        \abs{ g(t) } \leq \frac{ 1 }{ 7 } \text{ para todo } t \in \R
    ,\]
    e \[
        \norm{ 
            h_{ m } - \sum_{ k=1 }^{ 5 } g \circ \Phi_{ k }
         }_{ \infty } < \frac{ 7 }{ 8 }
    .\]
    Com isso, \[
        \norm{ 
            f - \sum_{ k=1 }^{ 5 } g \circ \Phi_{ k }
         }_{ \infty } \leq
        \norm{ f - h_{ m } }_{ \infty }
        + \norm{ 
            h_{ m } - \sum_{ k=1 }^{ 5 } g \circ \Phi_{ k }
         }_{ \infty }
         < \frac{ 1 }{ 72 } + \frac{ 7 }{ 8 }
         = \frac{ 8 }{ 9 }
    ,\]
    o que termina a prova.
\end{proof}
Agora podemos enunciar a demonstração do teorema \ref{teo: hilbert13}.
\begin{proof}[Demonstração do teorema \ref{teo: hilbert13}]
    Pelo lema \ref{lem: baire}, podemos fixar \( \lambda \in \R \) e \( \varphi_{ 1 }, \dots, \varphi_{ 5 } \in C(\I) \) tais que, dada \( f \in C(\I^2) \), existe \( g_{ 0 } \in C(\R) \) satisfazendo
    \begin{equation}
        \abs{ g_{ 0 }(t) } \leq \frac{ 1 }{ 7 } \norm{ f }_{ \infty } \text{ para todo } t \in \R
        \label{eq: g}
    ,\end{equation}
    e
    \begin{equation}
        \norm{ 
            f -
            \sum_{ k=1 }^{ 5 } g_{ 0 } \circ \Phi_{ k }
         }_{ \infty }
         < \frac{ 8 }{ 9 } \norm{ f }_{ \infty }
         \label{eq: f-(phi o g)}
    .\end{equation}
    Defina \( f_{ 0 } \defeq f \). Então, supondo definidas \( f_{ 0 }, \dots, f_{ n } \), \( g_{ 0 }, \dots, g_{ n } \), ponha \( f_{ n+1 } = f_{ n } - \sum_{ k=1 }^{ 5 } g_{ n } \circ \Phi_{ k } \).
    Logo, existe \( g_{ n+1 } \) satisfazendo (\ref{eq: g}) e (\ref{eq: f-(phi o g)}) com \( g_{ n+1 } \) no lugar de \( g_{ 0 } \) e \( f_{ n+1 } \) no lugar de \( f \).
    
    Dessa forma, temos \( \norm{ g_{ n } }_{ \infty } \leq \frac{ 1 }{ 7 } \norm{ f_{ n } }_{ \infty } \) e \( \norm{ f_{ n+1 } }_{ \infty } = \norm{ f_{ n } - \sum_{ k=1 }^{ 5 } g_{ n } \circ \Phi_{ k } }_{ \infty } \leq \frac{ 8 }{ 9 } \norm{ f_{ n } }_{ \infty } \).
    Portanto, \[
        \norm{ f_{ n } }_{ \infty } \leq \left( \frac{ 8 }{ 9 } \right)^{ n } \norm{ f_{ 0 } }_{ \infty } = \left( \frac{ 8 }{ 9 } \right)^{ n } \norm{ f }_{ \infty }
    ,\]
    e \[
        \norm{ g_{ n } }_{ \infty } <
        \frac{ 1 }{ 7 } \left( \frac{ 8 }{ 9 } \right)^{ n } \norm{ f }
    .\]
    Sendo assim, a série \( \sum g_{ n } \) converge em módulo para uma certa \( g \in C(\R) \) e, com isso, \[
        f = \sum_{ n\in\N } f_{ n } - f_{ n+1 }
        = \sum_{ n\in\N } \sum_{ k=1 }^{ 5 } g_{ n } \circ \Phi_{ k }
        = \sum_{ k=1 }^{ n } \left( 
            \sum_{ n\in\N } g_{ n }
         \right)
         \circ \Phi_{ k }
        = \sum_{ k=1 }^{ 5 } g \circ \Phi_{ k }
    ,\]
    o que conclui a demonstração.
\end{proof}