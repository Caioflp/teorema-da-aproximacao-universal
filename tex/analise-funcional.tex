% !TeX root = ./main.tex

\section{Elementos de Análise Funcional}
\label{app: func anal}

Aqui apresentaremos noções elementares que mesclam conceitos de Análise e de Álgebra Linear.

\begin{defn}
    Dado um espaço vetorial \( V \) sobre um corpo \( \K \), uma \emph{norma} em \( V \) é uma função \( \norm{ \cdot } : V \to [0, +\infty) \) tal que
    \begin{enumerate}[label=\roman*)]
        \item \( \norm{ v } = 0 \) se, e somente se \( v = 0 \) (normas são \emph{positivas definidas});
        \item \( \norm{ \alpha v } = \abs{ \alpha } \norm{ v } \) para todo \( v \in V \) e \( \alpha \in \K \) (\emph{homogeneidade absoluta});
        \item \( \norm{ v + w } \leq \norm{ v } + \norm{ w } \) para todos \( v, w \in V \) (\emph{desigualdade triangular}).
    \end{enumerate}
\end{defn}
Uma função que cumpre as propriedades \emph{ii} e \emph{iii} acima é denominada uma \emph{semi-norma}.
Um espaço vetorial junto de sua norma e denominado um \emph{espaço vetorial normado}.
A menos de onde houver ambiguidade, utilizaremos o mesmo símbolo \( \norm{ \cdot } \) para nos referir a normas de qualquer espaço vetorial normado.
O vetor dentro do símbolo deixará claro a norma de qual espaço está sendo utilizada.

É interessante perceber que todo espaço vetorial normado é um espaço métrico, quando se introduz nele a métrica induzida pela norma: \( (v, w) \mapsto \norm{ v - w } \).
As propriedades de uma métrica são facilmente verificadas.

\begin{defn}
    Uma transformação linear \( T : V \to W \) entre dois espaços vetoriais normados é dita \emph{limitada} se existe \( C \in [0, +\infty) \) tal que \( \norm{ Tv } \leq C \norm{ v } \) para todo \( v \in V \).
\end{defn}
Essa definição naturalmente é diferente da definição usual de função limitada, pois, como \( T(\lambda v) = \lambda (Tv) \), impossível termos \( \norm{ Tv } \leq C \) para todo \( v \in V \) se \( T \) é não nula.

O próximo teorema ilustra a utilidade dessa definição de transformação limitada.

\begin{teo}
    Seja \( T : V \to W \) uma transformação linear entre espaços vetoriais normados.
    As seguintes afirmativas são equivalentes:
    \begin{enumerate}[label=\roman*)]
        \item \( T \) é contínua;
        \item \( T \) é contínua em \( 0 \);
        \item \( T \) é limitada.
    \end{enumerate}
\end{teo}
\begin{proof}
    A implição de \emph{i} para \emph{ii} é óbvia.
    Suponha, agora, que \( T \) é contínua em \( 0 \).
    Tomando \( \varepsilon = 1 \), conseguimos \( \delta > 0 \) tal que se \( \norm{ v } \leq \delta \), então \( \norm{ Tv } < 1 \).
    Agora, dado \( v \in V \), \( v \neq 0 \), tome o vetor \( v' \defeq \delta v / \norm{ v } \).
    Claramente temos \( \norm{ v' } = \delta \), ou seja, \( \norm{ Tv' } < 1 \), o que implica \( \norm{ Tv } < \delta^{ -1 } \norm{ v } \).
    Logo, tomando \( C = \delta^{ -1 } \), \( T \) é limitada.
    Por fim, se \( T \) é limitada, dados \( a \in V \) e \( \varepsilon \), tome \( \delta < \varepsilon/C \).
    Com isso se \( \norm{ x - a } < \delta \) temos: \[
        \norm{ Tx - Ta } = \norm{ T(x - a) } \leq C \norm{ x - a } < C \cdot \varepsilon/C = \varepsilon
    .\]
    Portanto, \( T \) é contínua.
\end{proof}

Denotamos pro \( L(V, W) \) o conjunto das transformações lineares limitadas de \( V \) em \( W \).
Esse conjunto, com as operações usuais de adição e multiplicação por escalar, é um espaço vetorial.
Podemos, ainda, definir nele uma norma, dada por
\begin{align}
    \norm{ T } &\defeq \sup \left\{ \frac{ \norm{ Tv } }{ \norm{ v } } : v \in V \backslash \left\{ 0 \right\} \right\} \label{ali: first} \\
    &= \sup \left\{ \norm{ Tv } : v \in V, \norm{ v } = 1 \right\} \label{ali: sec} \\
    &= \inf \left\{ C : \norm{ Tv } \leq C \norm{ v } \text{ para todo } v \in V \right\}
    \label{ali: third}
.\end{align}
Claramente os dois primeiros conjuntos são iguais, logo seus supremos também o são.
Para ver que vale (\ref{ali: third}), perceba que, por (\ref{ali: first}), temos \( \norm{ Tv } \leq \norm{ T } \norm{ v } \) para todo \( v \in V \).
Portanto, sendo \( A \) o conjunto em (\ref{ali: third}), temos \( \norm{ T } \in A \), ou seja, \( \inf A \leq \norm{ T } \).
Além disso, dado \( C \in A \), temos \( C \geq \norm{ Tv } / \norm{ v } \) para todo \( v \in V - \left\{ 0 \right\} \).
Logo, \( \norm{ T } \leq C \) e, com isso, \( \norm{ T } \leq \inf A \).
Sendo assim, vale (\ref{ali: third}).
As propriedades i), ii) e iii) decorrem diretamente das propriedades das normas de \( V \) e \( W \) e são facilmente verificadas.
Sempre trataremos o conjunto \( L(V, W) \) como um espaço vetorial normado munido dessa norma, a qual é denominada \emph{norma de operador}. 
No caso especial em que \( W = \K \),
o espaço \( L(V, \K) \) é denominado \emph{espaço dual} de \( V \) e é denotado por \( V^{ * } \).
Os elementos de \( V^{ * } \) são denominados \emph{funcionais lineares}.

\begin{defn}
    Um \emph{produto interno} no espaço vetorial \( V \) é um funcional \( (v, w) \mapsto \dotprod{v,w} \), de \( V \times V \) em \( \K \), de modo que, para todos \( u, v, w \in V \) e \( \alpha \in \K \) valha
    \begin{enumerate}[label=\roman*)]
        \item \( \dotprod{\alpha v + u, w} = \bar{\alpha} \dotprod{v, w} + \dotprod{u, w} \);
        \item \( \dotprod{v, w} = \overline{\dotprod{v,w}} \);
        \item \( \dotprod{v, v} \geq \) e \( \dotprod{v, v} = 0 \) se, e somente se, \( v = 0 \).
    \end{enumerate}
    O par \( (V, \dotprod{\cdot,\cdot}) \) é chamado, então, de \emph{espaço com produto interno} ou \emph{espaço pré-Hilbertiano}.
\end{defn}
Como de costume, \( \bar{\alpha} \) denota o complexo conjugado de \( \alpha \).
Em um espaço vetorial com produto interno, a menos que se dia o contrário, tem-se \( \norm{ v } = \sqrt{ \dotprod{v,v} } \).
De fato essa função define uma norma em \( V \), como será mostrado a seguir.
Antes, porém, devemos estabelecer um resultado fundamental em espaços com produto interno.
\begin{teo}[Desigualdade de Cauchy-Schwarz]
    Em um espaço com produto interno \( (V, \dotprod{\cdot,\cdot}) \), tem-se \[
        \abs{ \dotprod{v, w} } \leq \norm{ v } \norm{ w }
    \]
    para todos \( v, w \in V \), com a igualdade ocorrendo se, e somente se, \( v = \lambda w \).
\end{teo}
\begin{proof}
    Se \( \dotprod{v, w} = 0 \), a igualdade é imediata.
    Caso contrário, então \( w \neq 0 \) e, para \( \lambda \in \K \) tem-se \[
        0 \leq \dotprod{v - \lambda w, v - \lambda w} = \norm{ v }^2 -\lambda \dotprod{v, w} -\bar{\lambda} \dotprod{w, v} + \abs{ \lambda }^2 \norm{ w }^2
    .\]
    tomando \( \lambda = \dotprod{w, v}/\norm{ w }^2 \) vem \[
        0 \leq \norm{ v }^2 - \abs{ \dotprod{v, w} }^2/ \norm{ w }^2
    ,\]
    o que implica \[
        \abs{ \dotprod{v, w} }^2 \leq \norm{ v }^2 \norm{ w }^2
    ,\]
    de onde o resultado se segue.
    Naturalmente a igualdade ocorre se e somente se existe \( \lambda \) tal que \( \dotprod{v - \lambda w, v - \lambda w} = 0 \), ou seja, \( v = \lambda w \).
\end{proof}

É fácil ver que as duas primeiras propriedades da norma são satisfeitas por \( \norm{ \cdot } \) como definida anteriormente.
Para verficar a desigualdade triangular perceba que
\begin{align*}
    \norm{ v + w }^2
    &= \norm{ v }^2 + 2 \Re \dotprod{v, w} + \norm{ w }^2 \\
    &\leq \norm{ v }^2 + 2 \abs{ \dotprod{v, w} } + \norm{ w }^2 \\
    &\leq \norm{ v }^2 + 2 \norm{ v } \norm{ w } + \norm{ w }^2 \\
    &= ( \norm{ v } + \norm{ w } )^2
.\end{align*}
Portanto, \( \norm{ \cdot } \) de fato é uma norma, denominada \emph{norma induzida pelo produto interno}.

\begin{defn}
    Um \emph{espaço de Hilbert} é um espaço com produto interno completo com relação à norma induzida pelo produto interno.
\end{defn}

\begin{defn}
    Um espaço vetorial \( V \) é a \emph{soma direta} de dois de seus subespaços \( W_{ 1 } \) e \( W_{ 2 } \) se todo \( v \in V \) possui uma representação única \[
        v = w_{ 1 } + w_{ w }
    ,\]
    com \( w_{ 1 } \in W_{ 1 } \) e \( w_{ 2 } \in W_{ 2 } \).
    Nesse caso, escrevemos \[
        V = W_{ 1 } \oplus W_{ 2 }
    .\]
\end{defn}
\begin{rem}
    Naturalmente, dados os subespaços disjuntos \( W_{ 1 } \) e \( W_{ 2 } \), podemos também construir o espaço vetorial \( W_{ 1 } \oplus W_{ 2 } \).
\end{rem}