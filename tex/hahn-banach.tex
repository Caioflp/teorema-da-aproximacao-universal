% !TeX root = ./main.tex
\section{Teorema de Hahn-Banach}

Um dos dois teoremas essenciais para a prova do \uat \ é o Teorema de Hahn-Banach.
Em Análise Funcional ele é relevante por garantir a existência de funcionais lineares limitados em um espaço vetorial normado qualquer.
Nossa exposição é inteiramente baseada em \cite{func-anal} e o leitor que quiser revisar os conceitos de Análise Funcional necessários para compreendê-la poderá encontrá-los no Apêndice \ref{app: func anal}.

\begin{defn}
    Se \( V \) é um espaço vetorial real, um \emph{funcional sublinear} em \( V \) é uma função \( p : V \to \R \) tal que \[
        p ( x + y ) \leq p(x) + p(y) \text{ e } p(\lambda x) = \lambda p(x) \text{ para todos \( x, y \in V \) e \( \lambda \in \R_{ \geq 0 } \)}
    .\]
\end{defn}

Um exemplo de funcional sublinear é uma seminorma em \( V \).
Observe que nada se exige sobre o sinal de \( p \).

\begin{teo}[Hahn-Banach]
    \label{thm: hahn-banach}
    Sejam \( V \) um espaço vetorial real, \( M \) um subespaço de \( V \) e \( p \) um funcional sublinear em \( V \).
    Se \( f \) é um funcional linear em \( M \) dominado por \( p \), ou seja, tal que \( f(v) \leq p(v) \) para todo \( v \in M \), então existe um funcional linear \( F \) em \( V \), que coincide com \( f \) em \( M \) e que também é dominado por \( p \).
    O funcional \( F \) é dito \emph{extensão de Hahn-Banach} de \( f \).
\end{teo}

Em outras palavras, se, em um espaço vetorial, temos um funcional sublinear que domina um funcional linear definido em um subespaço, podemos extender esse funcional ao espaço todo, mantendo a relação de dominância.

Antes de apresentar a demonstração do Teorema \ref{thm: hahn-banach}, vamos introduzir alguns conceitos de Teoria dos Conjuntos.

%% Escrever sobre o teorema, dar uma intuição da importância dele

\subsection{Lema de Zorn}

Dados conjuntos \( X \) e \( Y \), uma \emph{relação} de \( X \) em \( Y \) é um subconjunto \( R \) de \( X \times Y \). 
Escreveremos \( x R y \) para indicar que \( ( x, y ) \in R \).
Perceba que uma função \( f : X \to Y \) é uma relação de \( X \) em \( Y \) dada por \( ( x, y ) \in f \) se, e somente se, \( y = f(x) \).
Quando \( R \subset X \times X \), diremos que \( R \) é uma relação em \( X \).

Uma \emph{relação de ordem parcial} em \( X \) é uma relação que satisfaz, dados \( x, y, z \in X \):
\begin{enumerate}[label=\roman*)]
    \item \( x R x \) (\emph{reflexividade});
    \item Se \( x R y \) e \( y R x \), então \( x = y \) (\emph{antisimetria});
    \item Se \( x R y \) e \( y R z \), então \( x R z \) (\emph{transitividade}).
\end{enumerate}
O termo \emph{parcial} é utilizado para indicar que podem existir elementos de \( X \) não relacionados por essa ordem.
Caso tenhamos \( x R y \) ou \( y R x \) para todos \( x, y \in X \), então \( R \) é dita uma relação de ordem \emph{total}.
Utilizaremos o símbolo \( \prec \) para indicar uma ordem parcial.
Dizemos então que \( ( X, \prec ) \) é um conjunto parcialmente ordenado.
Se \( \prec \) também é total, \( ( X, \prec ) \) é totalmente ordenado.
Claramente todo subconjunto de \( X \) também é parcialmente ordenado, com a ordem induzida por \( \prec \).

Um \emph{elemento maximal} de um conjunto ordenado \( X \) é um elemento \( x \) tal que se \( y \in X \) com \( x \prec y \), então \( y = x \).
Uma \emph{cota superior} de um subconjunto \( Y \subset X \) é um elemento \( x \in X \) tal que \( y \prec x \) para todo \( y \in Y \).
Observe que se \( \prec \) não é total, não necessariamente um elemento maximal de \( Y \subset X \) é uma cota superior de \( Y \).

Dois exemplos usuais de conjuntos ordenados são \( \R \) e \( \wp ( X ) \) (o conjunto das partes de \( X \)), para um dado conjunto \( X \).
O primeiro é um conjunto totalmente ordenado pela ordem usual \( \leq \), o qual não possui elemento maximal.
O segundo se torna um conjunto ordenado ao dizermos que dados \( A, B \subset X \), temos \( A \prec B \) se, e somente se \( A \subset  B \).
Essa ordem não é total e o único elemento maximal de \( \wp ( X ) \) é o próprio \( X \).

\begin{axiom}[Lema de Zorn]
    Todo conjunto parcialmente ordenado, tal que todos seus subconjuntos totalmente ordenados possuam cota superior, possui elemento maximal.
\end{axiom}

Encerramos essa apresentação com um resultado que exemplifica como o lema de Zorn geralmente é utilizado.

\begin{prop}
    Todo espaço vetorial não trivial, ou seja, que possui elementos não nulos, possui uma base.
\end{prop}
\begin{proof}
    Seja \( V \) o espaço vetorial em questão, e \( A \) a coleção de todos os subconjuntos linearmente independentes de \( V \).
    Claramente essa coleção é parcialmente ordenada pela relação de inclusão.
    Desejamos mostrar, pelo Lema de Zorn, que \( A \) possui um elemento maximal, o qual será uma base de \( V \).
    Para tanto, tomamos uma subcoleção \( B \subset A \) totalmente ordenada e vamos mostrar que ela possui uma cota superior.
    Consideramos, então, o conjunto \[
        C = \bigcup_{ \beta \in B } \beta
    .\]
    Obviamente \( C \) é ``maior'' que todos os elementos de \( B \) (na ordem dada pela relação de inclusão).
    Para que \( C \) seja uma cota superior de \( B \), resta mostrar que ele é um conjunto linearmente independente.
    De fato, dado um conjunto finito \( \left\{ x_{ 1 }, \dots, x_{ n } \right\} \subset C \), sejam \( \beta_{ 1 }, \dots, \beta_{ n } \) os conjuntos de \( B \) tais que \( x_{ i } \in \beta_{ i } \) para todo \( i \leq n \).
    Como \( B \) é totalmente ordenado, existe \( \beta_{ k } \) tal que \( \beta_{ i } \subset \beta_{ k } \) para todo \( i \leq n \).
    Logo \( \left\{ x_{ i }, \dots, x_{ n } \right\} \subset  \beta_{ k } \) e, como \( \beta_{ k } \) é linearmente independente, também o são os \( x_{ i } \).
    Dessa forma, \( C \) é um conjunto linearmente independente.

    Portanto, aplicando o Lema de Zorn obtemos um elemento maximal \( \Lambda \) de \( A \).
    Denotando por \( \lin \Lambda \) o conjunto formado pelas combinações lineares de elementos de \( \Lambda \), afirmamos que \( \lin \Lambda = A \).
    Com efeito, supondo, por absurdo, que exista \( v \in V \backslash \lin \Lambda \), temos que \( \Lambda \cup \left\{ v \right\} \) é linearmente independente, ou seja, pertence a \( A \) e contém \( \Lambda \), um contradição pois \( \Lambda \) é maximal.
\end{proof}

\subsection{Demonstração do teorema de Hahn-Banach}

Seja \( \left\{ g_{ \lambda } \right\}_{ \lambda \in L } \), para algum conjunto de índices \( L \), a coleção das extensões lineares de \( f \) em conjuntos \( M_{ \lambda } \) tais que \( M \subset  M_{ \lambda } \subset V \), que são dominadas por \( p \).
Como \( f \in \left\{ g_{ \lambda } \right\}_{ \lambda \in L } \), essa coleção é não-vazia.
Podemos ordená-la parcialmente utilizando a relação de inclusão, considerando que \( g_{ \lambda } \) é um subconjunto de \( M_{ \lambda } \times \R \).
Em outras palavras, diremos que \( g_{ \alpha } \subset g_{ \beta } \) se tivermos \( M_{ \alpha } \subset M_{ \beta } \) e \( g_{ \beta } \mid_{ M_{ \alpha } } = g_{ \alpha } \), ou seja, se \( g_{ \beta } \) for uma extensão de \( g_{ \alpha } \).
Desejamos mostrar que existe um elemento maximal de \( \left\{ g_{ \lambda } \right\}_{ \lambda \in L } \).
Para tanto, vamos aplicar o Lema de Zorn.
Dado \( L' \subset L \) tal que \( \left\{ g_{ \lambda } \right\}_{ \lambda \in L' } \) é uma subcoleção totalmente ordenada, definimos \( g \) como \[
    g \defeq \bigcup_{ \lambda \in L' } g_{ \lambda }
,\]
ou seja, \( g \) é um funcional linear em \( \bigcup_{ \lambda \in L } M_{ \lambda } \) tal que \( g(v) = g_{ \lambda } (v) \), onde \( v \in M_{ \lambda } \).
De fato \( g \) está bem definida, pois se \( v \in M_{ \lambda_{ 1 } } \cap M_{ \lambda_{ 2 } } \) com \( \lambda_{ 1 } \neq \lambda_{ 2 } \), ambos pertencentes a \( L' \), então podemos supor, sem perda de generalidade, que \( g_{ \lambda_{ 2 } } \subset g_{ \lambda_{ 1 } } \) e, assim, \( g_{ \lambda_{ 1 } } (v) = g_{ \lambda_{ 2 } }(v) \), para \( v \in M_{ \lambda_{ 1 } } \cap M_{ \lambda_{ 2 } } = M_{ \lambda_{ 2 } } \). 
É fácil verificar que \( g \) estende \( f \) e é dominado por \( p \).
Logo, \( g \) é uma cota superior de \( \left\{ g_{ \lambda } \right\}_{ \lambda \in L' } \) e, pelo Lema de Zorn, existe \( F \in \left\{ g_{ \lambda } \right\}_{ \lambda \in L } \), definido em \( W \subset  V \), maximal.

Intuitivamente é claro que \( F \) deve estar definida em todo o espaço \( V \), de modo que é a extensão de Hahn-Banach de \( f \).
Para demonstrar esse fato, suponha, por absurdo, que ele seja falso, ou seja, que exista \( \eta \in V \backslash W \).
Nossa estratégia será construir um uma extensão \( G \) de \( F \), dominada por \( p \), definida em \( U = \lin (W \cup \left\{ \eta \right\}) \), contradizendo a maximalidade de \( F \).

Para definirmos \( G \), basta atribuirmos um valor para \( G(\eta) \).
De fato, todo elemento de \( U \) é da forma \( w + \alpha \eta \), onde \( w \in W \) e \( \alpha \in \R \).
Logo, pondo \[
    G(w + \alpha \eta) = F(w) + \alpha G(\eta)
,\]
claramente \( G \) é uma extensão linear de \( F \).
O passo crucial é definir \( G(\eta) \) de forma que \( G \) seja dominada por \( p \).
Para tanto, vamos nos atentar a algumas desigualdades.
Dados \( w_{ 1 }, w_{ 2 } \in W \), vale \[
    F(w_{ 1 }) + F(w_{ 2 }) =
    F(w_{ 1 } + w_{ 2 }) \leq
    p(w_{ 1 } + w_{ 2 }) \leq
    p(w_{ 1 } - \eta) + p(\eta + w_{ 2 })
.\]
Equivalentemente: \[
    F(w_{ 1 }) - p(w_{ 1 } - \eta) \leq p(\eta + w_{ 2 }) - F(w_{ 2 })
.\]
Ou seja, temos \[
    \sup \left\{ F(w_{ 1 }) - p(w_{ 1 } - \eta) \tq w_{ 1 } \in W \right\} \leq
    \inf \left\{ p(\eta + w_{ 2 }) - F(w_{ 2 }) \tq w_{ 2 } \in W \right\}
.\]
Defina, então, \( G(\eta) \) de modo que \[
    \sup \left\{ F(w_{ 1 }) - p(w_{ 1 } - \eta) \tq w_{ 1 } \in W \right\} \leq
    G(\eta) \leq
    \inf \left\{ p(\eta + w_{ 2 }) - F(w_{ 2 }) \tq w_{ 2 } \in W \right\}
.\]
Dessa forma, dado \( w + \alpha \eta \in U \), se \( \alpha > 0 \) temos
\begin{align*}
    G(w + \alpha \eta) &=
    F(w) + \alpha G(\eta) \\
    &\leq F(w) + \alpha \left( 
        p \left( \frac{ w }{ \alpha } + \eta \right) - F \left(  \frac{ w }{ \alpha }  \right)
    \right) \\
    &= F(w) + p(w + \alpha \eta) - F(w) \\
    &= p(w + \alpha \eta)
.\end{align*}
Caso \( \alpha < 0 \):
\begin{align*}
    G(w + \alpha \eta) &=
    F(w) + \alpha G(\eta) \\
    &\leq F(w) + \alpha \left( 
        F \left( \frac{ w }{ \abs{ \alpha } } \right) -
        p \left( \frac{ w }{ \abs{ \alpha } } - \eta \right)
    \right) \\
    &= F(w) - F(w) - p(-w - \alpha \eta) \\
    &= p(w + \alpha \eta)
.\end{align*}
Se \( \alpha = 0 \), \( G \) coincide com \( F \) e claramente é dominada por \( p \).
Sendo assim, chegamos a uma contradição com a maximalidade de \( F \), o que conclui a prova. \qed

No caso especial em que \( p \) é uma seminorma, a condição \( f \leq p \) é equivalente a \( \abs{ f } \leq p \), pois \( -f(x) = f(-x) \leq p(-x) = p(x) \).

\subsection{Aplicações do Teorema de Hahn-Banach}

Aqui vamos utilizar o Teorema de Hahn-Banach para provar um resultado sobre subespaços vetoriais densos que será utilizado na prova do teorema da aproximação universal.
Começamos apresentando uma aplicação do Teorema de Hahn-Banach para ilustrar como ele é geralmente utilizado.

\begin{teo}
    Seja \( V \) um espaço vetorial normado.
    Então:
    \begin{enumerate}[label=\roman*)]
        \item Se \( 0 \neq v \in V \), então existe \( f \in V^{ * } \) tal que \( f(v) = \norm{ v } \) e \( \norm{ f } = 1 \);
        \item Se \( v \) e \( w \) são elementos distintos de \( V \), então existe \( f \in V^{ * } \) tal que \( f(v) \neq f(w) \);
        \item Se \( v \in V \) satisfaz \( f(v) = 0 \) para todo \( f \in V^{ * } \), então \( v = 0 \);
        \item Se \( v \in V \), então \[
            \norm{ v }
            = \sup_{ 0 \neq f \in V^{ * } } \frac{ \abs{ f(v) } }{ \norm{ f } }
            = \max_{ 0 \neq f \in V^{ * } } \frac{ \abs{ f(v) } }{ \norm{ f } }
        .\]
    \end{enumerate}
\end{teo}
\begin{proof}
    Para provar (\emph{i}), basta aplicar Hahn-Banach com o funcional sublinear dado pela norma, o subespaço \( M = \lin (\left\{ v \right\}) \) e o funcional \( f : M \to \R \) dado por \( f(\alpha v) = \alpha \norm{ v } \).
    Sendo \( F \) a extensão de Hahn-Banach de \( f \), temos \( F(v) = \norm{ v } \) e, como o funcional sublinear é a norma de \( V \), vale \( \abs{ F(x) } \leq \norm{ x } \) para todo \( x \in V \).
    Como há pelo menos um caso de igualdade, vale \( \norm{ F } = 1 \).
    Em (\emph{ii}), como \( v \neq w \), então \( v - w \neq 0 \).
    Por (\emph{i}), isso implica a existência de um funcional \( f \) tal que \( f(v - w) = \norm{ v - w } \neq 0 \), ou seja, \( f(v) \neq f(w) \).
    Claramente (\emph{iii}) segue diretamente de (\emph{ii}).
    Como, para todo \( f \in V^{ * } \), vale \( \abs{ f(v) } \leq \norm{ f } \norm{ v } \), temos \[
        \norm{ v } \geq \sup_{ 0 \neq f \in V^{ * } } \frac{ \abs{ f(v) } }{ \norm{ f } }
    .\]
    Por (\emph{i}), existe \( g \in V^{ * } \) tal que \( g(v) = \norm{ v } \) e \( \norm{ g } = 1 \), de modo que \[
        \norm{ v } = \frac{ \abs{ g(v) } }{ \norm{ g } }
    ,\]
    ou seja, vale a igualdade e podemos trocar \( \sup \) por \( \max \).
\end{proof}

A próxima proposição será útil para provar o principal resultado dessa seção.
\begin{prop}
    Seja \( V \) um espaço vetorial normado e \( M \) um subespaço próprio fechado de \( V \).
    Dado \( v \in V \backslash M \), definimos \( \delta = d(v, M) \defeq \inf \left\{ \norm{ v - w } : w \in M \right\} \).
    Então existe um funcional \( f \in V^{ * } \) tal que \( \norm{ f } = 1 \), \( f(v) = \delta \) e \( f  \mid_{ M } = 0 \).
    \label{prop: closed_subset}
\end{prop}
\begin{proof}
    Como \( M \) é fechado, \( \delta > 0 \).
    Defina o funcional linear \( g : \lin ( \left\{ v \right\} \cup M ) \to \R \) dado por \[
        g(\alpha v + w) = \alpha \delta
    ,\]
    onde \( \alpha \in \R \) e \( w \in M \).
    Claramente temos \( g \mid_{ M } = 0 \) e \( g(v) = \delta \).
    Agora perceba que \( -w/\alpha \in M \) e, portanto, \[
        \abs{ g(\alpha v + w) } = \abs{ \alpha } \delta \leq \abs{ \alpha } \norm{ v + w / \alpha } = \norm{ \alpha v + w }
    ,\]
    de modo que \( \norm{ g } \leq 1 \).
    Além disso, para todo \( w \in M \) vale \[
        \norm{ g } \geq \frac{ \abs{ g(v + w) } }{ \norm{ v + w } } = \frac{ \delta }{ \norm{ v + w } }
    .\]
    Portanto, \[
        \norm{ g } \geq \sup_{ w \in M } \frac{ \delta }{ \norm{ v + w } } = \frac{ \delta }{ \inf_{ w \in M } \norm{ v - w } } = 1
    .\]
    Logo, \( \norm{ g } = 1 \).
    Então podemos aplicar Hahn-Banach, obtendo a extensão \( f \) de \( g \), utilizando como funcional sublinear a norma em \( V \).
\end{proof}

Agora o teorema que será usada na prova do \uat.

\begin{teo}
    Um subespaço \( M \) de um espaço vetorial \( V \) é denso se, e somente se, o único elemento de \( V^{ * } \) que se anula em \( M \) é o funcional nulo.
    \label{thm: hahn_banach_corolary}
\end{teo}
\begin{proof}
    Suponha que \( M \) é um subespaço denso de \( V \) e seja \( f \in V^{ * } \) um funcional que se anula em \( M \).
    Dado qualquer \( v \in V \backslash M \), seja \( ( x_{ n } )_{ n \in \N } \) tal que \( x_{ n } \in M \) para todo \( n \) e \( x_{ n } \to v \).
    Como \( f \) é uma transformação linear limitada, é contínua e, portanto, \( f(x_{ n }) \to f(v) \).
    Porém \( f(x_{ n }) = 0 \) para todo \( n \), ou seja, \( f(x_{ n }) \to 0 = f(v) \), de modo que \( f \equiv 0 \).

    Reciprocamente, se o único elemento de \( V^{ * } \) que se anula em \( M \) é o funcional nulo, então necessariamente \( V \) é denso.
    Caso contrário, \( \overline{M} \) seria um subespaço próprio fechado de \( V \) e, pela Proposição \ref{prop: closed_subset} existiria um funcional não nulo que se anula em \( \overline{M} \) e, consequentemente, em \( M \).
\end{proof}