\section{Teorema da Representação de Riesz}

Esse teorema possui diversas formulações.
Em geral, elas têm como objetivo representar os funcionais lineares contínuos em um dado espaço vetorial de uma maneira mais natural, associando-os a elementos daquele mesmo espaço ou de outro.

Apesar de a versão que abordaremos inicialmente não ser a utilizada na demonstração do teorema da aproximação universal, acreditamos que é importante apresentar esse teorema primeiro no contexto natural de espaços de Hilbert.

Novamente, nossa exposição desse resultado clássico da análise funcional foi formulada com base em \cite{func-anal} e o leitor que quiser recordar conceitos fundamentais poderá encontrálos no apêndice \ref{app: func anal}.

\subsection{Em espaços de Hilbert}

Da forma que será primeiramente enunciado, esse resultado trata da associação natural que existe entre um espaço de Hilbert \( H \) e o seu dual, \( H^{ * } \). Apesar de ser um fato de certa forma trivial para espaços de dimensão finita, sua demonstração não é tão óbvia para espaços de Hilbert em geral.
Para demonstrá-lo, precisamos, antes, de passar por três resultados preliminares.
O leitor poderá encontrar alguns dos conceitos de análise funcional utilizados no apêndice correspondente.

\begin{lem}
    Dados \( v, w \) pertencentes a \( (V, \dotprod{\cdot,\cdot}) \), tem-se que \( v \perp w \), ou seja, \( \dotprod{v,w} = 0 \), se, e somente se,
    \begin{equation}
        \norm{ v + \lambda w } \geq \norm{ v }
        \label{eq: ineq}
    .\end{equation}
    para todo \( \lambda \in \K \).
    \label{lem: perp_equiv}
\end{lem}
\begin{proof}
    Evidementemente temos \[
        0
        \leq \norm{ v + \lambda w }^2
        = \norm{ v }^2 + 2 \Re (\dotprod{v,w}) + \abs{ \lambda }^2 \norm{ w }^2
    .\]
    Se \( v \perp w \), então temos
    \begin{align*}
        \norm{ v + \lambda w }^2
        &= \norm{ v }^2 + \abs{ \lambda }^2 \norm{ w }^2 \\
        &\geq \norm{ v }^2
    ,\end{align*}
    de onde a desigualdade (\ref{eq: ineq}) segue.
    Reciprocamente, se vale (\ref{eq: ineq}) para todo \( \lambda \in \K \), em especial tomando \( \lambda = -\dotprod{w,v}/ \norm{ w }^2 \) e elevando ambos lados ao quadrado ficamos com \( 0 \leq - \abs{ \dotprod{v,w} }^2 \), o que implica \( v \perp w \).
\end{proof}

\begin{lem}[Lei do paralelogramo]
    Dados \( v, w \) pertencentes a \( (V, \dotprod{\cdot,\cdot}) \), tem-se \[
        \norm{ v + w }^2 + \norm{ v - w }^2 = 2 \norm{ v }^2 + 2 \norm{ w }^2
    .\]
\end{lem}

Como a demonstração desse lema consiste simplesmente em expandir o lado esquerdo da igualdade e usar as propridedades de produto interno, será deixada a cargo do leitor.

Antes do próximo resultado, uma definição.
Dado um subespaço \( E \) de um espaço com produto interno \( V \), definimos \[
    E^{ \perp } = \left\{ v \in V : \dotprod{v,w} = 0 \text{ para todo } w \in E \right\}
.\]

\begin{teo}[Projeção ortogonal]
    Se \( E \) é subespaço vetorial fechado de um espaço de Hilbert \( H \), então \[
        H = E \oplus E^{ \perp }
    .\]
    \label{teo: proj_ort}
\end{teo}
\begin{proof}
    Dado \( v \in H \), definimos \( \delta = \inf \left\{ \norm{ v - w } : w \in E \right\} \).
    Seja \( (w_{ n })_{ n\in\N } \) uma sequência de elementos de \( E \) tais que \( \norm{ v - w_{ n } } \to \delta \).
    Então, sendo \( k \) e \( \ell \) números naturais, aplicando a lei do paralelogramo para os vetores \( w_{ k } - v \) e \( w_{ \ell } - v \) obtemos: \[
        2 \norm{ w_{ k } - v }^2 + 2 \norm{ w_{ \ell } - v }^2
        = \norm{ w_{ k } + w_{ \ell } - 2v }^2 + \norm{ w_{ k } - w_{ \ell } }
    ,\]
    o que implica, remanejando e lembrando que \( (w_{ k } + w_{ \ell })/2 \in E \),
    \begin{align*}
        \norm{ w_{ k } - w_{ \ell } }
        &= 2 \norm{ w_{ k } - v }^2 + 2 \norm{ w_{ \ell } - v }^2 - 4 \norm{ (w_{ k } + w_{ \ell })/2 - v }^2 \\
        &\leq 2 \norm{ w_{ k } - v }^2 + 2 \norm{ w_{ \ell } - v }^2 - 4 \delta^2
    .\end{align*}
    Com isso, concluímos que \( (w_{ n }) \) é uma sequência de Cauchy.
    Como \( H \) é um espaço de Hilbert, temos que \( w_{ n } \to w \in E \), pois \( E \) é fechado e, pela continuidade da norma, temos \( \norm{ v - w } = \delta \).

    Intuitivamente, \( w \) é o elemento de \( E \) mais próximo de \( v \).
    Logo, é razoável esperar que ele seja a projeção ortogonal de \( v \) em \( E \).
    Para confirmar essa suspeita, devemos verificar que \( v - w \in E^{ \perp } \).
    De fato, para todo \( \lambda \in \K \) e todo \( u \in E \) temos \[
        \norm{ (v - w) + \lambda u }
        = \norm{ v + (-w + \lambda u) }
        \geq \delta
        = \norm{ v - w }
    .\]
    Portanto, pelo lema \ref{lem: perp_equiv}, concluímos que \( v - w \in E^{ \perp } \).
    Sendo assim, temos \( v = w + (v - w) \), onde \( w \in E \) e \( v - w \in E^{ \perp } \).
    Para mostrar a unicidade dessa decomposição, suponha que \( v = w_{ 1 } + u_{ 1 } = w_{ 2 } + u_{ 2 } \), onde \( w_{ 1 }, w_{ 2 } \in E \) e \( u_{ 1 }, u_{ 2 } \in E^{ \perp } \).
    Então \[
        w_{ 1 } - w_{ 2 } = u_{ 2 } - u_{ 1 } \in E \cap E^{ \perp }
    ,\]
    o que implica \( w_{ 1 } - w_{ 2 } = u_{ 2 } - u_{ 1 } = 0 \), ou seja, \( w_{ 1 } = w_{ 2 } \) e \( u_{ 1 } = u_{ 2 } \).
\end{proof}

Agora podemos prosseguir para o principal resultado dessa subseção.

\begin{teo}[Teorema da representação de Riesz em espaços de Hilbert]
    Dado um espaço de Hilbert H e seu dual \( H^{ * } \), a função
    \begin{align*}
        \gamma : &H \to H^{ * } \\
            &v \mapsto \gamma(v) = f_{ v }
    ,\end{align*}
    tal que \( f_{ v } = \dotprod{v,\cdot} \), é uma isometria antilinear e sobrejetiva em \( H^{ * } \).
\end{teo}
\begin{proof}
    Para mostrar que \( \gamma \) é uma isometria, provaremos que \( \norm{ f_{ v } } = \norm{ v } \) para todo \( v \in H \).
    De fato, se \( v = 0 \) isso é evidente.
    Fixado \( v \in H \backslash \left\{ 0 \right\} \), pela desigualdade de Cauchy-Schwarz temos \( \abs{ f_{ v }(w) } = \abs{ \dotprod{v,w} } \leq \norm{ v } \norm{ w } \).
    Logo, \( \norm{ f_{ v } } \leq \norm{ v } \).
    Por outro lado, temos \( \norm{ v }^2 = \dotprod{v,v} = f_{ v }(v) \leq \norm{ f_{ v } } \norm{ v } \), ou seja, \( \norm{ v } \leq \norm{ f_{ v } } \) e acabamos.

    Além disso claramente \( \gamma \) é antilinear, pois \[
        \gamma (\alpha v + w) = \dotprod{\alpha v + w, \cdot} = \bar{\alpha} \dotprod{v, \cdot} + \dotprod{w,\cdot} = \bar{\alpha} \gamma(v) + \gamma(w)
    .\]

    A parte menos óbvia, e que dá nome ao teorema, é a da sobrejetividade.
    Repare que, com essa propriedade, todo funcional \( f \in H^{ * } \) fica unicamente associado a um determinado \( v \in H \) tal que \( f = \gamma(v) \).
    Dizemos, então, que \( v \) \emph{representa} \( f \).
    Para demonstrá-la, utilizaremos o teorema \ref{teo: proj_ort}.

    Dado \( f \in H^{ * } \), se \( f = 0 \) então naturalmente \( f = \gamma(0) \).
    Se \( f \neq 0 \), repare que \( \ker(f) \), o núcleo de \( f \), é um subespaço próprio fechado de \( H \).
    Pelo teorema \ref{teo: proj_ort}, temos \[
        H = \ker(f) \oplus \ker(f)^{ \perp }
    ,\]
    de onde concluímos que existe \( w \in \ker(f)^{ \perp } \) satisfazendo \( \norm{ w } = 1 \).
    Agora uma observação simples, porém fundamental: para todos \( u, v \in H \) temos \( f(v)u - f(u)v \in \ker(f) \).
    Logo, para todo \( v \in H \) temos \[
        \dotprod{w,f(v)w - f(w)v} = 0
    ,\]
    o que implica, desenvolvendo, \[
        f(v) = \dotprod{\overline{f(w)}w,v}
    .\]
    Sendo assim, \( f = \gamma(\overline{f(w)}w) \).
\end{proof}

\subsection{Em teoria da medida}

No contexto de teoria da medida, existem alguns resultados que são reconhecidos como ``teorema da representação de Riesz''.
O mais conhecido estabelece que dual do espaço das funções cujo módulo é \( p \) integrável em um espaço de medida \( (\Omega, \Sigma, \mu) \), o espaço \( L_{ p } \), onde \( 1 < p < +\infty \), é representado pelo espaço \( L_{ q } \), onde \( 1/p + 1/q = 1 \).
Mais precisamente, se \( G : L_{ p } \to \R \) é um funcional linear limitado, então existe \( g \in L_{ q } \) tal que, para toda \( f \in L_{ p } \) temos \[
    G(f) = \int fg \ \mathrm{d}\mu
.\]
Se \( \mu \) é uma medida \( \sigma \)-finita, então o resultado também é válido para \( L_{ 1 } \), cujo espaço dual é \( L_{ \infty } \).
Observe que se \(  p = q = 2 \), então o espaço \( L_{ 2 } \) é um espaço de Hilbert e, então, o teorema da representação de Riesz se torna o mesmo apresentado na subseção anterior.

Outro teorema que também leva esse nome, e que é o que utilizaremos para provar o \uat , utiliza medidas para representar funcionais definidos num espaço de funções reais contínuas.
A versão exata de que precisamos tem uma demonstração que foge do escopo deste trabalho, portanto, antes de apresentá-la vamos enunciar e demonstrar uma versão mais simples, que trata dos funcionais definidos em \( C([a, b]) \).
Porém antes, uma definição:
\begin{defn}
    Dado um espaço \( F \) formado pelas funções de um conjunto qualquer \( X \) em \( \R \), um funcional linear \( G : F \to \R \) é dito \emph{positivo} se \( G(f) \geq 0 \) sempre que \( f \geq 0 \), ou seja, sempre que \( f(x) \geq 0 \) para todo \( x \in X \).
\end{defn}

\begin{rem}
    Perceba que se \( f \leq g \) para todo \( x \in X \), então \( g-f \geq 0 \), o que implica \( G(g-f) \geq 0 \) e, pela linearidade de \( G \), \( G(g) \geq G(f) \).
\end{rem}

\begin{teo}[Teorema da representação de Riesz]
    Seja \( J = [a, b] \).
    Se \( G \) é um funcional linear limitado positivo no espaço \( C(J) \), munido com a norma do supremo, então existe uma medida \( \gamma \), definida nos subconjuntos de Borel de \( \R \), tal que
    \begin{equation}
        G(f) = \int_{ J } f \ \mathrm{d}\gamma
        \label{eq: representation}
    .\end{equation}
    para toda \( f \in C(J) \).
    Além disso, \( \norm{ G } = \gamma(J) \).
\end{teo}
\begin{proof}
    A demonstração aqui apresentada envolve conceitos e resultados não-triviais de teoria da medida, os quais ocupariam muito espaço se fossem apresentados em um apêndice como fizemos até aqui.
    O leitor não familizarizado com esse assunto pode utilizar o capítulo 9 do livro de onde obtemos essa prova, \cite{bartle}, como referência.

    Começamos definindo uma classe de funções auxiliares que nos será útil.
    Para todo \( t \) tal que \( a \leq t < b \), seja \( \varphi_{ t,n } \) a função pertencente a \( C(J) \) que é igual a \( 1 \) em \( [a, t] \), é igual a \( 0 \) em \( \left(t+ \frac{ 1 }{ n }, b\right] \) e é linear em \( \left(t, t + \frac{ 1 }{ n }\right] \).
    Se \( n \leq m \) e \( x \in J \), então \( 0 \leq \varphi_{ t, m }(x) \leq \varphi_{ t, m }(x) \leq 1 \), de modo que, pela observação feita anteriormente, as sequências \( (G(\varphi_{ t, n }))_{ n\in\N } \) são monótonas não-crescentes e limitadas.
    Portanto, para cada \( t \in [a, b) \) podemos definir \[
        g(t) = \lim_{ n\to\infty } G(\varphi_{ t,n })
    .\]
    Além disso, pomos \( g(t) = 0 \) para \( t < a \) e, se \( t \geq b \), definimos \( g(t) = G(\varphi_{ b }) \), onde \( \varphi_{ b }(x) = 1 \) para todo \( x \in J \).
    Como \( \varphi_{ t_{ 0 }, n } \leq \varphi_{ t_{ 1 }, n } \) para \( t_{ 0 } \leq t_{ 1 } \), vê-se que \( g \) é uma função monótona não-decrescente em \( \R \).

    Nosso objetivo é utilizar o teorema da extensão de Hahn para definir uma medida \( \gamma \) nos subconjuntos de Borel de \( \R \), tal que \( \gamma((\alpha, \beta]) = g(\beta) - g(\alpha) \), porém, para tanto, precisamos antes demonstrar que \( g \) é contínua pela direita.
    Isso é claro se \( t < a \) ou se \( t \geq b \), porém não se \( t \in [a, b) \).
    Neste último caso, tome \( \varepsilon > 0 \) e seja \( n_{ 0 } \in \N \) suficientemente grande para que tenhamos
    \begin{equation}
        n_{ 0 } > \sup \left\{ 2, \frac{ \norm{ G } }{ \varepsilon } \right\}
        \label{eq: n0}
    .\end{equation}
    e, também,
    \begin{equation}
        g(t) \leq G(\varphi_{ t, n_{ 0 } }) \leq g(t) + \varepsilon
        \label{eq: g_sandwich}
    .\end{equation}

    Agora definimos uma nova função auxiliar.
    Seja \( \psi_{ n } \in C(J) \), \( n > 2 \), tal que \( \psi_{ n } \) é igual a \( 1 \) em \( \left[a, t + \frac{ 1 }{ n^2 }\right] \), é igual a \( 0 \) em \( \left(t + \frac{ 1 }{ n } - \frac{ 1 }{ n^2 }, b\right] \) e é linear em \[
        \left(t + \frac{ 1 }{ n^2 }, t + \frac{ 1 }{ n } - \frac{ 1 }{ n^2 }\right)
    .\]
    Um exercício mecânico de analisar os valores de \( \psi_{ n } \) e \( \varphi_{ t, n } \) nos pontos onde a diferença entre elas é máxima revela que \( \norm{ \psi_{ n } - \varphi_{ t, n } }_{ \infty } \leq \frac{ 1 }{ n } \).
    Logo,
    \begin{align*}
        G(\psi_{ n }) - G(\varphi_{ t, n })
        &\leq\abs{ G(\psi_{ n }) - G(\varphi_{ t, n }) } \\
        &= \abs{ G(\psi_{ n } - \varphi_{ t, n }) } \\
        &\leq \norm{ G } \norm{ \psi_{ n } - \varphi_{ t, n } }_{ \infty } \\
        &\leq \frac{ \norm{ G } }{ n }
    ,\end{align*}
    de onde obtemos \[
        G(\psi_{ n }) \leq G(\varphi_{ t, n }) + \frac{ \norm{ G } }{ n }
    \]
    para todo \( n \in \N \).
    Em especial, tomando \( n \geq n_{ 0 } \) ficamos com, devido a (\ref{eq: n0}) e (\ref{eq: g_sandwich}), \[
        G(\psi_{ n_{ 0 } }) \leq g(t) + 2 \varepsilon
    .\]
    Como \( g \) é monótona não-decrescente, temos \( g(t) \leq g \left(t + \frac{ 1 }{ n_{ 0 }^2 } \right) \).
    Além disso, se \( n \) é suficentemente grande é fácil ver que \[
        \varphi_{ t + \frac{ 1 }{ n_{ 0 }^2 }, n } \leq \psi_{ n_{ 0 } }
    ,\]
    o que implica \( g \left( t + \frac{ 1 }{ n_{ 0 }^2 } \right) \leq G(\psi_{ n_{ 0 } }) \).
    Com isso, concluímos que \[
        g(t) \leq g \left( t + \frac{ 1 }{ n_{ 0 }^2 } \right) \leq g(t) + 2 \varepsilon
    \]
    e, assim, \( g \) é contínua pela direita.

    Aplicando o teorema da extensão de Hahn, existe uma medida \( \gamma \), nos subconjuntos de Borel de \( \R \), tal que \( \gamma \left((\alpha, \beta]\right) = g(\beta) - g(\alpha) \).
    Em particular, isso implica que \( \gamma(E) = 0 \), se \( E \cap J = \emptyset \), que \[
        \gamma ([a, c]) = \gamma((a-1, a) \cup [a, c]) = \gamma((a-1, c]) = g(c)
    \]
    e que \( \norm{ G } = \abs{ G(\varphi_{ b }) } = g(b) = \gamma(J) \).

    Resta mostrar que, de fato, \( \gamma \) representa \( G \) no sentido de (\ref{eq: representation}).
    Pela continuidade uniforme de \( f \) em \( J \), fixamos \( \varepsilon > 0 \), e tomamos \( \delta > 0 \) tal que se \( \abs{ x - y } < \delta \) e \( x, y \in J \), então \( \abs{ f(x) - f(y) } < \varepsilon \).
    Agora seja \( a = t_{ 0 } < t_{ 1 } < \cdots < t_{ m } = b \) uma partição de \( J \) tal que \( \sup \left\{ t_{ k } - t_{ k-1 } \right\} < \frac{ \delta }{ 2 } \) e tome \( n \in \N \) grande o suficiente para que tenhamos \( 2/n < \inf \{t_{ k } - t_{ k-1 }\} \) e, para \( k = 1, \dots, m \),
    \begin{equation}
        g(t_{ k }) \leq G(\varphi_{ t_{ k }, n }) \leq g(t_{ k }) + \frac{ \varepsilon }{ m \norm{ f }_{ \infty } }
        \label{eq: g_sandwich_2}
    .\end{equation}
    Defina em \( J \), então, as funções
    \begin{align*}
        f_{ 1 }(x)
        &= \varphi_{ t_{ 1 }, n }(x) + \sum_{ k=2 }^{ m } f(t_{ k }) \left(
            \varphi_{ t_{ k }, n }(x) - \varphi_{ t_{ k-1 }n }(x)
        \right) \\
        f_{ 2 }(x)
        &= f(t_{ 1 }) \chi_{ [t_{ 0 }, t_{ 1 }] }(x) +
        \sum_{ k=2 }^{ m } f(t_{ k }) \chi_{ (t_{ k-1 }, t_{ k }] }(x)
    .\end{align*}
    Perceba que \( f_{ 1 } \in C(J) \) e que \( f_{ 2 } \) é uma função escada em \( J \).
    Está claro que \( \sup \left\{ \abs{ f_{ 2 }(x) - f(x) } : x \in J \right\} \leq \varepsilon \) e com um certo esforço é possível mostrar que \( \norm{ f_{ 1 } - f }_{ \infty } \leq \varepsilon \).
    Portanto, temos
    \begin{equation}
        \abs{ G(f) - g(f_{ 1 }) } \leq \varepsilon \norm{ G }
        \label{eq: triangle_1}
    .\end{equation}
    Rescrevendo (\ref{eq: g_sandwich_2}), temos \[
        0 \leq G(\varphi_{ t_{ k }, n }) - g(t_{ k }) \leq \frac{ \varepsilon }{ m \norm{ f }_{ \infty } }
    \]
    para todo \( k = 1, \dots, m \).
    Logo, se \( 2 \leq k \leq m \), temos
    \begin{align*}
        \abs{ G(\varphi_{ t_{ k }, n } - \varphi_{ t_{ k-1 }, n }) - (g_{ t_{ k } } - g_{ t_{ k-1 } })}
        &= \abs{ \left(
            G(\varphi_{ t_{ k }, n }) - g_{ t_{ k } }
        \right) -
        \left(
            G(\varphi_{ t_{ k-1 }, n } - g(t_{ k-1 }))
        \right)} \\
        &\leq \frac{ \varepsilon }{ m \norm{ f }_{ \infty } }
    .\end{align*}
    Aplicando \( G \) a \( f_{ 1 } \) e integrando \( f_{ 2 } \) com respeito a \( \gamma \), a desigualdade acima implica
    \begin{equation}
        \abs{ 
            G(f_{ 1 }) - \int_{ J } f_{ 2 } \ \mathrm{d}\gamma
            \leq \varepsilon
         }
        \label{eq: triangle_2}
    .\end{equation}
    Porém, como \( f_{ 2 } \) está a uma distância de no máximo \( \varepsilon \) de \( f \), também temos
    \begin{equation}
        \abs{ 
            \int_{ J } f_{ 2 } \ \mathrm{d}\gamma
            - \int_{ J } f \ \mathrm{d}\gamma
            \leq \varepsilon \gamma(J)
         }
        \label{eq: triangle_3}
    .\end{equation}
    Combinando (\ref{eq: triangle_1}), (\ref{eq: triangle_2}), (\ref{eq: triangle_3}) e utilizando a desigualdade triangular, obtemos \[
        \abs{ G(f) - \int_{ J } f \ \mathrm{d}\gamma }
        \leq \varepsilon (2 \norm{ G } + 1)
    .\]
    Como \( \varepsilon \) é arbitrário, temos (\ref{eq: representation}).
\end{proof}

\begin{rem}
    Como é demonstrado no capítulo 9 de \cite{bartle}, se \( G \) for um funcional linear limitado qualquer em \( C(J) \), podemos decompô-lo como \( G = G^{ + } - G^{ - } \), onde \( G^{ + } \) e \( G^{ - } \) são funcionais lineares limitados positivos.
    Sendo \( \gamma \) e \( \nu \) as medidas que representam \( G^{ + } \) e \( G^{ - } \) no sentido de (\ref{eq: representation}), respectivamente, se definirmos a medida com sinal \( \mu = \gamma - \nu \), então teremos \[
        G(f) = \int_{ J } f \ \mathrm{d}\mu
    \]
    para toda \( f \in C(J) \).
\end{rem}

Terminamos essa seção com uma versão mais geral desse teorema, que utilizaremos na prova do \uat.
O leitor interessado em uma demonstração pode consultar a seção 13.5 de \cite{royden}.

\begin{Riesz}
    Seja \( X \) um espaço compacto Hausdorff e \( C(X) \) o espaço das funções reais contínuas em \( X \).
    Então, a cada funcional linear limintado \( F \) em \( C(X) \) corresponde uma única medida de Baire, finita, com sinal, \( \nu \) em \( X \) tal que \[
        F(f) = \int f \ \mathrm{d}\nu
    \]
    para cada \( f \) em \( C(X) \).
    Além disso, \( \norm{ F } = \abs{ \nu }(X) \).
\end{Riesz}

\begin{rem}
    Utilizaremos esse teorema com \( X = [0, 1]^{ n } \).
    Como nesse caso \( X \) é um espaço métrico compacto, a medida \( \nu \) será uma medida de Borel.
\end{rem}