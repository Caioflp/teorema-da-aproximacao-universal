\documentclass[15pt]{beamer}

%% Packages

\usepackage{mathtools, amsthm}
\usepackage{exercise-sheet}
\usepackage{global-macros}
\usepackage{tikz}
\usetikzlibrary{positioning}
\usepackage[
    backend=biber,
    style=alphabetic,
    sorting=ynt
]{biblatex}
\usepackage{csquotes}
\usepackage{hyperref}

%% Environments

\theoremstyle{plain} % default
\newtheorem{teo}{Teorema}[section]
\newtheorem{lem}{Lema}[section]
\newtheorem{prop}{Proposição}[section]
\newtheorem{cor}[teo]{Corolário}
\newtheorem*{axiom}{Axioma}

\newtheorem*{TAU}{Teorema da Aproximação Universal}
\newtheorem*{Riesz}{Teorema da Representação de Riesz}

\theoremstyle{definition}
\newtheorem{defn}{Definição}[section]
\newtheorem{conj}{Conjectura}[section]
\newtheorem{exmp}{Exemplo}[section]
\newtheorem{rem}{Observação}[section]

\theoremstyle{remark}
\newtheorem*{note}{Nota}
\newtheorem{case}{Caso}


% Macros

\renewcommand{\vec}[1]{\mathbf{#1}}
\renewcommand{\Re}{\text{Re}}

\newcommand{\K}{\mathbb{K}}
\newcommand{\I}{\mathbb{I}}
\newcommand{\uat}{\textbf{TAU}}

\DeclarePairedDelimiter{\dotprod}{\langle}{\rangle}

\DeclareMathOperator{\rk}{rk}
\DeclareMathOperator{\intt}{int}
\DeclareMathOperator{\diam}{diam}
\DeclareMathOperator{\rref}{rref}
\DeclareMathOperator{\vspan}{span}
\DeclareMathOperator{\proj}{proj}
\DeclareMathOperator{\lin}{Lin}
\DeclareMathOperator{\supp}{supp}

%% Numbering

\numberwithin{equation}{section}


\title[Eigenfaces]{Reconhecimento Facial utilizando Eigenfaces}
\author[C. Lins, J. Martins]{Caio Lins \and Juliane Martins}
\institute[EMAp]{FGV - EMAp}
\date[EMAp 2020]{2020}
\logo{\includegraphics[height=0.8cm]{fgv_logo.png}}


\begin{document}

\frame{\titlepage}

\begin{frame}
    \frametitle{Introdução}

    \begin{itemize}
        \item<1-> Como humanos reconhecem faces?

        \item<2-> Reconhecimento por Eigenfaces (PCA).
    \end{itemize}

\end{frame}


\begin{frame}
    \frametitle{Reconhecimento utilizando Eigenfaces}

    \begin{itemize}
        \item<1-> Dados utilizados: ``AT\&T Database of Faces''\cite{att}.
        \( 40 \) indivíduos, \( 10 \) imagens de cada.
        \item<2-> Conjunto de treino: \( 9 \) das \( 10 \) fotos de \( 39 \) indivíduos, totalizando \( 351 \) imagens \( 112 \times 92 \) em escala de cinza.
         
    \end{itemize}

\end{frame}

\begin{frame}
    \frametitle{Reconhecimento utilizando Eigenfaces}

    \begin{itemize}
        \item Equivalente a \( \left\{ \Gamma_{ i } \in \R^{ 10304 } \ ; \ 1 \leq i \leq 351 \right\} \)
    \end{itemize}
    
    \pause\bigskip

    \[
        \begin{bmatrix}
            0.336 & \cdots & 0.947 \\
            \vdots & \ddots & \vdots \\
            0.214 & \cdots & 0.445
        \end{bmatrix}_{ 112\times92 }
        \longrightarrow
        \begin{bmatrix}
            0.336 \\
            \vdots \\
            0.947 \\
            \vdots \\
            0.214 \\
            \vdots \\
            0.445
        \end{bmatrix}_{ 10304\times1 }
    .\]
\end{frame}

\begin{frame}
    \frametitle{Reconhecimento utilizando Eigenfaces}

    \begin{itemize}
        \item<1-> Cálculo da \emph{face média}: \[
                \Psi = \frac{ 1 }{ 351 } \sum_{ i=1 }^{ 351 } \Gamma_{ i }
            .\]

        \item<2-> Centralização dos dados: \[
                \Phi_{ i } = \Gamma_{ i } - \Psi
            .\]

    \end{itemize}

    \pause

    \begin{center}
        \includegraphics[height=4cm]{average-face.jpg}
    \end{center}

\end{frame}

\begin{frame}
    \frametitle{Reconhecimento utilizando Eigenfaces}

    \begin{itemize}
        \item<1-> Agora define-se a matriz de dados: \[
                A =
                \begin{bmatrix}
                    \Phi_{ 1 } & \cdots & \Phi_{ 351 }
                \end{bmatrix}
            .\]

        \item<2-> Componentes principais por SVD:
            Matrizes \( U_{ 10304 \times 351 }, V_{ 351 \times 351 } \) ortogonais e \( \Sigma_{ 351 \times 351 } \) diagonal tais que \[
                A = U \Sigma V^{ T }
            .\]

        \item<3-> Colunas de \( U \leftrightarrow \) autovetores de \( \frac{ 1 }{ n-1 } AA^{ T } \) (matriz de covariância).
            Elas são as \emph{Eigenfaces}.

    \end{itemize}
\end{frame}

\begin{frame}
    \frametitle{Reconhecimento utilizando Eigenfaces}

    \begin{itemize}
        \item Eigenfaces:
            \begin{center}
                \begin{figure}
                    \includegraphics[width=.3\textwidth]{eigenface1}
                    \includegraphics[width=.3\textwidth]{eigenface2}
                    \\[\smallskipamount]
                    \includegraphics[width=.3\textwidth]{eigenface348}
                    \includegraphics[width=.3\textwidth]{eigenface349}
                    \caption{Eigenfaces com maiores valores singulares (em cima) e menores valores singulares (embaixo).}
                \end{figure}
            \end{center}
    \end{itemize}
\end{frame}

\begin{frame}
    \frametitle{Reconhecimento utilizando Eigenfaces}

    \begin{itemize}
        \item Decaimento dos valores singulares:
            \begin{figure}
                \begin{center}
                    \includegraphics[width=.6\textwidth]{singular-values.png}
                    \caption{Percebe-se uma rápida queda nos valores singulares, para próximo de \( 0 \).
                    Utilizamos, então, as \( 33 \) primeiras eigenfaces.}
                \end{center}
            \end{figure}
    \end{itemize}

\end{frame} 

\begin{frame}
    \frametitle{Reconhecimento utilizando Eigenfaces}

    \begin{itemize}
        \item<1-> Projeta-se os vetores \( \Phi_{ i } \) no subespaço \( E \subset \R^{ 10304 } \) gerado pelas eigenfaces \( q_{ 1 }, \dots, q_{ 33 } \) (\emph{face space}).

        \item <2-> Para cada \( \Phi_{ i } \), obtém-se um vetor de coeficientes \( \Omega_{ i } \) tal que \[
                \proj_{ E } \Phi_{ i } =
                \begin{bmatrix}
                    q_{ 1 } & \cdots & q_{ 33 }
                \end{bmatrix}
                \Omega_{ i }
            .\]

        \item<3-> Para a imagem desconhecida \( \Gamma \), obtém-se, de maneira análoga, os vetores \( \Phi \) e \( \Omega \).
    \end{itemize}
\end{frame}

    \begin{frame}
        \frametitle{Reconhecimento utilizando Eigenfaces}

        \begin{center}
            \begin{figure}
                \includegraphics[width=.7\textwidth]{diagrama.png}
                \caption{Representação da projeção de algumas imagens no face space\cite{turk}.}
            \end{figure}
        \end{center}
    
    \end{frame}

\begin{frame}
    \frametitle{Reconhecimento utilizando Eigenfaces}

    \begin{itemize}
        \item<1-> Caso o erro de projeção \( \norm{ \Phi - \proj_{ E } \Phi } \) seja maior que um limite \( \theta_{ \delta } \), a imagem não é uma face.
    \end{itemize}

\end{frame}

\begin{frame}
    \frametitle{Reconhecimento utilizando Eigenfaces: Algoritmo 1}

    \begin{itemize}
        \item<1-> Os vetores \( \Omega_{ i } \) são agrupados por indivíduo (\( 9 \) vetores para cada um).
        \item<2-> Em cada grupo, é calculada a média dos valores \( \norm{ \Omega - \Omega_{ i } } \).
        \item<3-> A foto tem mais chances de pertencer ao indivíduo que tiver menor média associada.
    \end{itemize}

\end{frame}

\begin{frame}
    \frametitle{Reconhecimento utilizando Eigenfaces: Algoritmo \( 2 \)}

    \begin{itemize}
        \item<1-> Calcula-se o erro \( \norm{ \Omega - \Omega_{ i } } \) para todas as imagens do conjunto de treino.
        \item<2-> Identifica-se qual imagem apresenta o menor erro.
        \item<3-> A foto desconhecida tem mais chances de pertencer ao indivíduo presente nessa imagem.
        % \item<1-> Imagem desconhecida tem mais chances de pertencer ao indivíduo presente na foto que minimiza a quantia \( \norm{ \Omega - \Omega_{ i } } \), onde \( i \) varia de \( 1 \) a \( 351 \).
    \end{itemize}

\end{frame}

\begin{frame}
    \frametitle{Reconhecimento utilizando Eigenfaces}

    \begin{itemize}
        \item<1- > Nos dois Algoritmos, se o mínimo obtido é maior que um limite \( \theta_{ \varepsilon } \), a imagem pertence a uma pessoa desconhecida.
            Esse limite é diferente para cada algoritmo.
        \item<2-> Escolha de \( \theta_{ \delta } \) e \( \theta_{ \varepsilon } \).
    \end{itemize}

\end{frame}

\begin{frame}
    \frametitle{Resultados obtidos}

    A técnica foi testada com três tipos de imagem:
    \begin{itemize}
        \item Tipo \( A \) (\( 351 \) imagens): Imagem estava no conjunto de treino.
        \item Tipo \( B \) (\( 39 \) imagens): Imagem não estava no conjunto de treino, mas a pessoa nela sim.
        \item Tipo \( C \) (\( 10 \) imagens): Pessoa da imagem não estava no conjunto de treino.

    \end{itemize}
\end{frame}

\begin{frame}
    \frametitle{Resultados obtidos: Algoritmo \( 1 \)}
    \begin{table}
        \begin{tabular}{ |c|c|c| }
            \hline
            Tipo Imagem & Não é face & Face desconhecida \\
            \hline
            \( A \) & \( 0.00 \) \% & \( 25.36 \) \% \\
            \hline
            \( B \) & \( 0.00 \) \% & \( 38.46 \) \% \\
            \hline
            C & \( 0.00 \) \% & \( 100.00 \) \% \\
            \hline
        \end{tabular}
    \end{table}

    \begin{table}
        \begin{tabular}{|c|c|c|}
            \hline
            Tipo Imagem & Reconheceu certo & Reconheceu errado \\
            \hline
            A & \( 73.79 \) \% & \( 0.85 \) \% \\
            \hline
            B & \( 61.54 \) \% & \( 0.00 \) \% \\
            \hline
            C & - & \( 0.00 \) \% \\
            \hline
        \end{tabular}
    \end{table}
\end{frame}

\begin{frame}
    \frametitle{Resultados obtidos: Algoritmo \( 2 \)}
    \begin{table}
        \begin{tabular}{ |c|c|c| }
            \hline
            Tipo Imagem & Não é face & Face desconhecida \\
            \hline
            \( A \) & \( 0.00 \) \% & \( 0.00 \) \% \\
            \hline
            \( B \) & \( 0.00 \) \% & \( 17.95 \) \% \\
            \hline
            C & \( 0.00 \) \% & \( 100.00 \) \% \\
            \hline
        \end{tabular}
    \end{table}

    \begin{table}
        \begin{tabular}{|c|c|c|}
            \hline
            Tipo Imagem & Reconheceu certo & Reconheceu errado \\
            \hline
            A & \( 100.00 \) \% & \( 0.00 \) \% \\
            \hline
            B & \( 82.05 \) \% & \( 0.00 \) \% \\
            \hline
            C & - & \( 0.00 \) \% \\
            \hline
        \end{tabular}
    \end{table}
\end{frame}

\begin{frame}
    \frametitle{Discussão dos resultados}

    \begin{itemize}
        \item<1-> Expectativa: Algoritmo \( 1 \) melhor para fotos do tipo \( B \) e Algoritmo \( 2 \), para fotos do tipo \( A \).
        \item<2-> Superioridade do Algoritmo \( 2 \).
        \item<3-> Principais fatores que afetam capacidade de reconhecimento:
            \begin{itemize}
                \item Tamanho do rosto na foto
                \item Posição do rosto na imagem
                \item Inclinação do rosto
                \item Iluminação
            \end{itemize}
    \end{itemize}

\end{frame}

\begin{frame}
    \frametitle{Discussão dos resultados}

    \begin{figure}
        \includegraphics[width=.17\textwidth]{s1_1}\hfill
        \includegraphics[width=.17\textwidth]{s1_2}\hfill
        \includegraphics[width=.17\textwidth]{s1_6}\hfill
        \includegraphics[width=.17\textwidth]{s1_9}
        \caption{Quatro imagens do indivíduo \( 1 \).}
        \label{fig: ind1}
    \end{figure}

\end{frame}

\begin{frame}
\frametitle{Referências}
    \bibliographystyle{plain}
    \bibliography{eigenfaces}
\end{frame}

\end{document}
