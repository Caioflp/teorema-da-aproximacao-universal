% !TeX root = ./main.tex

\section{Teorema da Aproximação Universal}

Nesta seção apresentaremos o principal resultado estudado: o teorema da aproximação universal, como formulado em \cite{cybenko89}.
Como o leitor pode lembrar, quando abordamos o \( 13^{ \circ } \) problema de Hilbert, enunciamos o caso geral e provamos um caso particular do teorema da superposição de Kolmogorov, segundo o qual podemos representar de forma \emph{exata} funções contínuas de \( \I^{ n } \) em \( \R \) utilizando a superposição e composição de funções contínuas de \( \I \) em \( \R \), de uma maneira parecida como a apresentada em (\ref{eq: neural_func_form}).
Entretanto, como comentado em \cite{cybenko89}, a diferença crucial entre esses dois resultados é que no teorema de Kolmogorov nos é permitido usar uma classe muito maior de não-linearidades, as quais variam de acordo com a função a ser aproximada.
Isso fornece uma intuição do porquê obter representações exatas é uma tarefa intratável no caso de redes neurais.

\subsection{O Teorema}

Denotaremos o espaço das funções reais contínuas em \( \I^{ n } \) por \( C(\I^{ n }) \) e o espaço das medidas de Borel finitas, com sinal e regulares em \( \I^{ n } \) por \( M(\I^{ n }) \).
Queremos entender sob quais condições as somas da forma \[
    G(x) = \sum_{ j=1 }^{ N } \alpha_{ j } \sigma(y_{ j }^{ T }x + \theta_{ j })
\]
são densas em \( C(\I^{ n }) \) com respeito à norma do supremo.
Faremos isso primeiro para uma classe mais geral de funções \( \sigma \) e depois mostraremos que as sigmoides, ou seja, funções \( \sigma \) tais que \[
    \sigma(x) \to
    \begin{cases}
        1, \text{ se } x \to +\infty \\
        0, \text{ se } x \to -\infty
    \end{cases}
,\]
pertencem a essa classe.
Essa classe de funções é importante pois as funções de ativação de redes neurais (as não-linearidades em cada nó) são, em geral, sigmoides.

\begin{defn}
    Dizemos que \( \sigma \) é \emph{discriminatória} se, para uma medida \( \mu \in M(I^{ n }) \), termos \[
        \int_{ \I^{ n } } \sigma(y^{ T }x + \theta)  \ \mathrm{d}\mu = 0
    \]
    para todos \( y \in \R^{ n } \) e \( \theta \in \R \) implica em \( \mu = 0 \).
\end{defn}

\begin{TAU}
    Seja \( \sigma \) uma função contínua discriminatória qualquer.
    Então as somas finitas da forma
    \begin{equation}
        G(x) = \sum_{ j=1 }^{ N } \alpha_{ j } \sigma(y_{ j }^{ T }x + \theta_{ j })
        \label{eq: neural_func}
    .\end{equation}
    são densas em \( C(\I^{ n }) \).
    Em outras palavras, dada qualquer \( f \in C(\I^{ n }) \) e \( \varepsilon > 0 \), existe uma soma, \( G(x) \), da forma acima, tal que \[
        \norm{ G - f }_{ \infty } < \varepsilon
    .\]
\end{TAU}

\begin{proof}
    Seja \( S \subset C(\I^{ n }) \) o conjunto das funções descritas por (\ref{eq: neural_func}).
    Claramente ele é um subespaço de \( C(\I^{ n }) \), o qual nós afirmamos ser denso.
    Suponha, por absurdo, que ele não o seja, isto é, se \( R \) é o fecho de \( S \), suponha que tenhamos \( C(\I^{ n }) \backslash \neq \emptyset \).
    Então, pelo teorema \ref{thm: hahn_banach_corolary}, existe um funcional linear limitado \( L : C(\I^{ n }) \to \R \), não nulo, tal que \( L(R) = 0 \).
    Pelo teorema da representação de Riesz, esse funcional é da forma \[
        L(h)
        = \int_{ \I^{ n } } h(x) \ \mathrm{d}\mu(x)
    ,\]
    para alguma \( \mu \in M(\I^{ n }) \) e toda \( h \in C(\I^{ n }) \).
    Em particular, como \( \sigma(y^{ T }x + \theta) \in S \) para todos \( y \in \R^{ n } \) e \( \theta \in \R \), temos \[
        \int_{ \I^{ n } } \sigma(y^{ T }x + \theta) \ \mathrm{d}\mu(x) = 0
    \]
    para todos \( y, \theta \).
    Porém, como \( \sigma \) é, por hipótese, discriminatória, isso implica \( L \) ser o funcional nulo, onde chegamos a uma contradição.
    Portanto, \( R = C(\I^{ n }) \), ou seja, \( S \) é denso em \( C(\I^{ n }) \).
\end{proof}

Tendo provado o teorema para funções discriminatórias, nos resta mostrar que, de fato, sigmoides são discriminatórias.

\begin{lem}
    Todas sigmoides limitadas e mensuráveis são discriminatórias.
\end{lem}

\begin{proof}
    Seja \( \sigma \) uma sigmoide e suponha que, para uma dada medida \( \mu \in M(\I^{ n }) \), tenhamos \[
        \int_{ \I^{ n } } \sigma(y^{ T }x + \theta) \ \mathrm{d}\mu(x)
    \]
    para todos \( y \in \R^{ n } \) e \( \theta \in \R \).
    Desejamos provar que isso implica \( \mu = 0 \).

    Começamos a demonstração reparando que, para quaisquer \( x, y, \theta, \varphi \) temos \[
        \sigma(\lambda(y^{ T }x + \theta) + \varphi)
        \to
        \begin{cases}
            1, \text{ para } y^{ T }x + \theta > 0 \text{ quando } \lambda \to +\infty \\
            0, \text{ para } y^{ T }x + \theta < 0 \text{ quando } \lambda \to +\infty \\
            \sigma(\varphi), \text{ para } y^{ T }x + \theta = 0 \text{ para todo } \lambda \\
        \end{cases}
    .\]
    Logo, a família de funções \( \sigma_{ \lambda }(x) = \sigma(\lambda(y^{ T }x + \theta) + \varphi) \) é limitada e converge pontualmente à função \[
        \gamma(x) =
        \begin{cases}
            1 \text{ para } y^{ T }x + \theta > 0 \\
            0 \text{ para } y^{ T }x + \theta < 0 \\
            \sigma(\varphi) \text{ para } y^{ T }x + \theta = 0 \\
        \end{cases}
    \]
    quando \( \lambda \to +\infty \).

    Seja \( \Pi_{ y, \theta } \) o hiperplano definido por \( \left\{ x \in \R^{ n } : y^{ T }x + \theta = 0 \right\} \) e seja \( H_{ y, \theta } \) o meio-espaço aberto definido por \( \left\{ x \in \R : y^{ T }x + \theta > 0 \right\} \).
    Então, pelo teorema da convergência dominada de Lebesgue, temos %% entender melhor essa aplicação do TCD
    \begin{align*}
        0 &=
        \int_{ \I^{ n } } \sigma_{ \lambda }(x)  \ \mathrm{d}\mu(x) \\
        &= \int_{ \I^{ n } } \gamma(x) \ \mathrm{d}\mu(x) \\
        &= \sigma(\varphi)\mu(\Pi_{ y, \theta }) + \mu(H_{ y, \theta })
    \end{align*}
    para todos \( \varphi, \theta \) e \( y \).

    %% Conferir com yuri
    Fazendo \( \varphi \to +\infty \), concluímos que \[
        0 = \mu(\Pi_{ y, \theta }) + \mu(H_{ y, \theta })
    \]
    para todos \( y, \theta \).
    Devemos mostrar agora que isso implica que isso implica a medida \( \mu \) ser, em si, nula.
    Observe que a dificuldade vem do fato de que \( \mu \) não é positiva.

    Fixe \( y \).
    Vamos definir um funcional linear \( F \) em \( L_{ \infty }(\R) \) como \[
        F(h) \defeq \int_{ I^{ n } } h(y^{ T }x) \ \mathrm{d}\mu(x)
    .\]
    Note que, como \( \mu \) é finita, \( \abs{ F(h) } \leq \abs{ \mu }(\I^{ n }) \norm{ h }_{ \infty } \) e, portanto, \( F \) é um funcional limitado.
    Agora seja \( h \) a função indicadora do intervalo \( [\theta, \infty) \), de modo que \[
        F(h) = \int_{ I^{ n } } h(y^{ T }x) \ \mathrm{d}\mu
        = \mu(\Pi_{ y, -\theta }) + \mu(H_{ y, -\theta }) = 0
    .\]
    De maneira análoga, \( F(h) = 0 \) se \( h \) é a função indicadora do intervalo aberto \( (\theta, \infty) \).
    Por linearidade, \( F(h) = 0 \) para a função indicadora de qualquer intervalo e, por tanto, para qualquer função simples (soma de funções indicadoras de intervalos).
    Como funções simples são densas em \( L_{ \infty }(\R) \), pela continuidade de \( F \) concluímos que \( F = 0 \).

    Em particular, as funções limitadas mensuráveis \( s_{ m }(u) = \sen(m^{ T }u) \) e \( c(u) = \cos(m^{ T }u) \) revelam que \[
        F(s_{ m } + i c_{ m })
        = \int_{ \I^{ n } } \cos(m^{ T }x) + i \sen(m^{ T }x) \ \mathrm{d}\mu(x)
        = \int_{ \I^{ n } } e^{ i m^{ T }x } \ \mathrm{d}\mu(x)
        = 0
    \]
    para todo \( m \in \R^{ n } \).
    Portanto, a transformada de Fourrier de \( \mu \) é \( 0 \) e , portanto, \( \mu \) deve ser a medida nula (o leitor pode consultar o corolário 8.4 de \cite{fourier}).
    Ou seja, \( \sigma \) é discriminatória.
\end{proof}


