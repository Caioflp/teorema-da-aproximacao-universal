\section{Teorema da Representação de Riesz}

O teorema da representação de Riesz, da forma que será enunciado, trata da associação natural que existe entre um espaço de Hilbert \( H \) e o seu dual, \( H^{ * } \). Apesar de ser um fato de certa forma trivial para espaços de dimensão finita, sua demonstração não é tão óbvia para espaços de Hilbert em geral.

Esse teorema possui diversas formulações.
Apesar de não ser a que apresentaremos inicialmente a utilizada na demonstração do teorema da aproximação universal, acreditamos que é importante apresentá-lo primeiro no contexto mais geral de espaços de Hilbert.

Entretanto, antes de enunciá-lo e demonstrá-lo, precisamos de dois resultados preliminares.

\begin{lem}
    Num espaço com produto intero, tem-se que \( v \perp w \), ou seja, \( \dotprod{v,w} = 0 \), se, e somente se,
    \begin{equation}
        \norm{ v + \lambda w } \geq \norm{ v }
        \label{eq: ineq}
    ,\end{equation}
    para todo \( \lambda \in \K \).
\end{lem}
\begin{proof}
    Evidementemente temos
    \begin{equation}
        0
        \leq \norm{ v + \lambda w }^2
        = \norm{ v }^2 + 2 \Re (\dotprod{v,w}) + \abs{ \lambda }^2 \norm{ w }^2
    .\end{equation}
    Se \( v \perp w \), então temos
    \begin{align}
        \norm{ v + \lambda w }^2
        &= \norm{ v }^2 + \abs{ \lambda }^2 \norm{ w }^2 \\
        &\geq \norm{ v }^2
    ,\end{align}
    de onde a desigualdade (\ref{eq: ineq}) segue.
    Reciprocamente, se vale (\ref{eq: ineq}) para todo \( \lambda \in \K \), em especial tomando \( \lambda = -\dotprod{w,v}/ \norm{ w }^2 \) e elevando ambos lados ao quadrado ficamos com \( 0 \leq - \abs{ \dotprod{v,w} }^2 \), o que implica \( v \perp w \).
\end{proof}

Antes do próximo resultado, uma definição.
Dado um subespaço \( E \) de um espaço com produto interno \( V \), definimos \[
    E^{ \perp } = \left\{ v \in V : \dotprod{v,w} = 0 \text{ para todo } w \in E \right\}
.\]

\begin{teo}[Projeção Ortogonal]
    Se \( E \) é subespaço vetorial fechado de um espaço de Hilbert \( H \), então 
    \begin{equation}
        H = E \oplus E^{ \perp }
    .\end{equation}
\end{teo}
\begin{proof}
    Dado \( v \in H \), definimos \( d(v, E) = \inf \left\{ \norm{ v - w } : w \in E \right\} \).
    %% TODO: Falar da lei do paralelogramo
\end{proof}