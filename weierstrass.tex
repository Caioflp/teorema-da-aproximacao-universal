% !TeX root = ./master.tex

\section{Teorema da aproximação de Weierstrass}

Desejamos mostrar que dada uma função contínua \( f : [a, b] \to \R \), podemos aproximá-la arbitrariamente bem por funções polinomiais \( p : [a, b] \to \R \).

Em outras palavras, seja \( C( [a, b]) \) o espaço das funções contínuas em \( [a, b] \). Indicamos por \( \norm{ \varphi }_{ \infty } \) a norma do supremo de uma função limitada \( \varphi : [a, b] \to \R \), ou seja, \[
    \norm{ \varphi }_{ \infty } = \sup \left\{ \abs{ \varphi(x) } \tq x \in [a, b] \right\}
.\]

Então é verdade que

\begin{teo}
    \label{thm: weierstrass}
    Dada \( f \in C( [a, b] ) \), para todo \( \varepsilon > 0 \) existe um polinômio \( p : [a, b] \to \R \) tal que \[
        \norm{ f - p }_{ \infty } < \varepsilon
    .\]
\end{teo}

Inicialmente, observamos que basta provar o teorema para o caso \( f \in C([0, 1]) \).
De fato, dada \( f \in C([a, b]) \), considere o homeomorfismo \( \varphi : [0, 1] \to [a, b] \) dado por \( \varphi(x) = a + ( b - a )x \), cuja inversa é \( \varphi^{ -1 } : [a, b] \to [0, 1] \) dada por \( \varphi^{ -1 } (x) = \frac{ x - a }{ b - a } \).
Então a função \( g = f \circ \varphi \) pertence a \( C([0, 1]) \) e, dado \( \varepsilon > 0 \), se existe um polinômio \( p(x) \) com \( \norm{ g - p }_{ \infty } < \varepsilon \), temos também, como \( \varphi^{ -1 } \) é um polinômio de grau \( 1 \), \[
    \norm{ g \circ \varphi^{ -1 } - p \circ \varphi^{ -1 } }_{ \infty } < \varepsilon
.\]
Como \( g \circ \varphi^{ -1 } = f \) e \( p \circ \varphi^{ -1 } \) é um polinômio, o resultado vale também para \( C([a, b]) \).

Em seguida, devemos definir a classe de polinômios que utilizaremos na demonstração.

\begin{defn}
    \label{def: bernstein}
    Dada \( g : X \to \R \) definimos o \( n \)-ésimo polinômio de Bernstein de \( g \) como
    \begin{equation}
        B_{ n }(x, g) \define
        \sum_{ k=0 }^{ n } g \left( \frac{ k }{ n } \right) \binom{n}{k} x^{ k } ( 1 - x )^{ n - k }
        \label{eq: bernstein}
    .\end{equation}
\end{defn}

Note a semelhança entre os polinômios de Bernstein e a expansão binomial de \( ( 1 + ( 1 - x ) )^{ n } \).
De fato, temos \( B_{ n } (x, 1) = ( 1 + ( 1 - x ) )^{ n } = 1 \).
Mais geralmente, para toda constante \( c \in \R \) tem-se \( B_{ n } (x, c) = c \).

Utilizaremos essa semelhança para obter algumas identidades essenciais para a demonstração do Teorema \ref{thm: weierstrass}.
Dados \( p \) e \( q \) reais, começamos considerando a expansão binomial de \( ( p + q )^{ n } \):\[
    ( p + q )^{ n } =
    \sum_{ k=0 }^{ n } \binom{n}{k} p^{ k } q^{ n-k }
.\]
Considerando ambos lados da igualdade como funções de \( p \), podemos derivá-los com relação a essa variável, obtendo \[
    n ( p + q )^{ n-1 } =
    \sum_{ k=0 }^{ n } k \binom{n}{k} p^{ k-1 } q^{ n-k }
.\]
Multiplicando ambos lados por \( p/n \), ficamos com
    \begin{equation}
        p ( p + q )^{ n-1 } =
        \sum_{ k=0 }^{ n } \frac{ k }{ n } \binom{n}{k} p^{ k } q^{ n-k }
        \label{eq: id_1}
    .\end{equation}
Essa é a primeira identidade, válida para todos \( p, q \in \R \).
Derivando novamente com relação a \( p \) e multiplicando ambos lados por \( p/n \) obtemos
\begin{equation}
    p^2 \left( 1 - \frac{ 1 }{ n } \right) ( p + q )^{ n-2 } + \frac{ p }{ n } ( p + q )^{ n-1 } =
    \sum_{ k=0 }^{ n } \frac{ k^2 }{ n^2 } \binom{n}{k} p^{ k } q^{ n-k }
    \label{eq: id_2}
,\end{equation}
a segunda identidade que utilizaremos.


Como consideramos \( f, g \in C([0, 1]) \), segue da Definição \ref{def: bernstein} que se \( f \geq 0 \), então \( B_{ n } (x, f) \geq 0 \) e, se \( f \leq g \), então \( B_{ n } (x, f) \leq B_{ n } (x, g) \).

Com essas ferramentas, podemos então apresentar a
%% Provar desigualdades envolvendo polinômios de Bernstein e explicar a restrição ao intervalo [0, 1]
 \begin{proof}[Demonstração do Teorema \ref{thm: weierstrass}]
    Observamos inicialmente que como \( f \) é uma função contínua definida em um compacto, é uniformemente contínua.
    Portanto, dado \( \varepsilon > 0 \), existe \( \delta > 0 \) tal que se \( x, y \in [0, 1] \) satisfazem \( \abs{ x - y } < \delta \) então \[
        \abs{ f(x) - f(y) } < \frac{ \varepsilon }{ 2 }
    .\]
    Agora, definimos \( M \define \norm{ f }_{ \infty } \) e fixamos \( \xi \in [0, 1] \).
    Logo, se \( \abs{ x - \xi } \geq \delta \) temos \[
        \abs{ f(x) - f(\xi) } \leq 2M \leq 2M \left( \frac{ x - \xi }{ \delta } \right)^2
    .\]
    Combinando as duas últimas desigualdades, concluímos que para todo \( x \in [0, 1] \) vale
    \begin{equation}
        \abs{ f(x) - f(\xi) } \leq 2M \left( \frac{ x - \xi }{ \delta } \right)^2 + \frac{ \varepsilon }{ 2 }
        \label{eq: weierstrass_bound_1}
    .\end{equation}

    Vamos aproximar \( f \) pelos seus polinômios de Bernstein.
    Seja \( B_{ n }(x, f) \) o \( n \)-ésimo polinômio de Bernstein de \( f \), avaliado em \( x \).
    Então
    \begin{align}
        \abs{ B_{ n } (x, f) - f(\xi) }
        &= \abs{B_{ n } (x, f - f(\xi))} \\
        &\leq  B_{ n } \left( x, 2M \left( \frac{ x - \xi }{ \delta } \right)^2 + \frac{ \varepsilon }{ 2 } \right) \\
        &= \frac{ 2M }{ \delta^2 } B_{ n } (x, ( x - \xi )^2) + \frac{ \varepsilon }{ 2 } \\
        &+ \frac{ 2M }{ \delta^2 } \left( B_{ n } (x, x^2) + B_{ n } (x, -2 x \xi) + \xi^2 \right) + \frac{ \varepsilon }{ 2 } \\
        &= \frac{ 2M }{ \delta^2 } \left( B_{ n } (x, x^2) - 2 \xi B_{ n } (x, x) + \xi^2 \right) + \frac{ \varepsilon }{ 2 }
        \label{al: elem_bernstein}
    .\end{align}
    Aqui fizemos uso das propriedades de \( B_{ n } (x, f) \) que seguem de \( x \in [0, 1] \), discutidas anteriormente.
    Utilizando as equações (\ref{eq: id_1}) e (\ref{eq: id_2}), com a substituição \( p = x \) e \( q = 1 - x \), concluímos que \[
        B_{ n } (x, x) = x
    \]
    e que \[
        B_{ n } (x, x^2) = x^2 \left( 1 - \frac{ 1 }{ n } \right) + \frac{ x }{ n }
    .\]
    Substituindo em (\ref{al: elem_bernstein}), ficamos com
    \begin{align}
        \frac{ 2M }{ \delta^2 } \left( B_{ n } (x, x^2) - 2 \xi B_{ n } (x, x) + \xi^2 \right) + \frac{ \varepsilon }{ 2 } &=
        \frac{ 2M }{ \delta^2 } \left( 
            x^2 \left( 1 - \frac{ 1 }{ n } \right) + \frac{ x }{ n }
            - 2 \xi x
            + \xi^2
         \right) + \frac{ \varepsilon }{ 2 } \\
         &= \frac{ 2M }{ \delta^2 } \left( 
             x^2 + \frac{ x - x^2 }{ n } - 2 \xi x + \xi^2
          \right) + \frac{ \varepsilon }{ 2 } \\
         &= \frac{ \varepsilon }{ 2 } + \frac{ 2M }{ n \delta^2 } ( x - x^2 ) + \frac{ 2M }{ \delta^2 } ( x - \xi )^2
    .\end{align}
    Sendo assim, 
    \begin{equation}
        \abs{ B_{ n } (x, f) - f(\xi) } \leq \frac{ \varepsilon }{ 2 } + \frac{ 2M }{ n \delta^2 } ( x - x^2 ) + \frac{ 2M }{ \delta^2 } ( x - \xi )^2
    .\end{equation}
    Como essa desigualdade vale para todo \( x \in [0, 1] \), em especial é válida para \( x = \xi \).
    Fazendo essa substituição, obtemos
    \begin{equation}
        \abs{ B_{ n } (\xi, f) - f(\xi) } \leq
        \frac{ \varepsilon }{ 2 } 
        + \frac{ 2M }{ n \delta^2 } ( \xi - \xi^2 )
    .\end{equation}
    Facilmente podemos verificar que \( \xi - \xi^2 \leq \frac{ 1 }{ 4 } \) para todo \( \xi \in [0, 1] \).
    Logo,
    \begin{equation}
        \abs{ B_{ n } (\xi, f) - f(\xi) } \leq
        \frac{ \varepsilon }{ 2 } 
        + \frac{ M }{ 2 n \delta^2 }
    .\end{equation}
    Por fim, tomando \( n > \frac{ M }{ \varepsilon \delta^2 } \), temos \( \frac{ M }{ 2 n \delta^2 } < \frac{ \varepsilon }{ 2 } \) e, assim,
    \begin{equation}
        \abs{ B_{ n } (\xi, f) - f(\xi) } < \varepsilon
    .\end{equation}
    Como o valor de \( n \) obtido para que essa desigualdade seja satisfeita depende apenas de \( \varepsilon \) (lembramos que \( \delta \) depende apenas de \( \varepsilon \), pela continuidade uniforme de \( f \)), ela é válida para todo \( \xi \in [0, 1] \), ou seja, \[
        \norm{ B_{ n } (\cdot , f) - f }_{ \infty } < \varepsilon
    .\qedhere\]
 \end{proof}
 

