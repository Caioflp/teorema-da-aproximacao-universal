\section{Teorema da Representação de Riesz}

Esse teorema possui diversas formulações.
Em geral, elas têm como objetivo representar os funcionais lineares contínuos em um dado espaço vetorial de uma maneira mais natural, associando-os a elementos daquele mesmo espaço ou de outro.

Apesar de a versão que abordaremos inicialmente não ser a utilizada na demonstração do teorema da aproximação universal, acreditamos que é importante apresentar esse teorema primeiro no contexto natural de espaços de Hilbert.

Novamente, nossa exposição desse resultado clássico da análise funcional foi formulada com base em \cite{func-anal} e o leitor que quiser recordar conceitos fundamentais poderá encontrálos no apêndice \ref{app: func anal}.

\subsection{Em espaços de Hilbert}

Da forma que será primeiramente enunciado, esse resultado trata da associação natural que existe entre um espaço de Hilbert \( H \) e o seu dual, \( H^{ * } \). Apesar de ser um fato de certa forma trivial para espaços de dimensão finita, sua demonstração não é tão óbvia para espaços de Hilbert em geral.
Para demonstrá-lo, precisamos, antes, de passar por três resultados preliminares.
O leitor poderá encontrar alguns dos conceitos de análise funcional utilizados no apêndice correspondente.

\begin{lem}
    Dados \( v, w \) pertencentes a \( (V, \dotprod{\cdot,\cdot}) \), tem-se que \( v \perp w \), ou seja, \( \dotprod{v,w} = 0 \), se, e somente se,
    \begin{equation}
        \norm{ v + \lambda w } \geq \norm{ v }
        \label{eq: ineq}
    .\end{equation}
    para todo \( \lambda \in \K \).
    \label{lem: perp_equiv}
\end{lem}
\begin{proof}
    Evidementemente temos \[
        0
        \leq \norm{ v + \lambda w }^2
        = \norm{ v }^2 + 2 \Re (\dotprod{v,w}) + \abs{ \lambda }^2 \norm{ w }^2
    .\]
    Se \( v \perp w \), então temos
    \begin{align*}
        \norm{ v + \lambda w }^2
        &= \norm{ v }^2 + \abs{ \lambda }^2 \norm{ w }^2 \\
        &\geq \norm{ v }^2
    ,\end{align*}
    de onde a desigualdade (\ref{eq: ineq}) segue.
    Reciprocamente, se vale (\ref{eq: ineq}) para todo \( \lambda \in \K \), em especial tomando \( \lambda = -\dotprod{w,v}/ \norm{ w }^2 \) e elevando ambos lados ao quadrado ficamos com \( 0 \leq - \abs{ \dotprod{v,w} }^2 \), o que implica \( v \perp w \).
\end{proof}

\begin{lem}[Lei do paralelogramo]
    Dados \( v, w \) pertencentes a \( (V, \dotprod{\cdot,\cdot}) \), tem-se \[
        \norm{ v + w }^2 + \norm{ v - w }^2 = 2 \norm{ v }^2 + 2 \norm{ w }^2
    .\]
\end{lem}

Como a demonstração desse lema consiste simplesmente em expandir o lado esquerdo da igualdade e usar as propridedades de produto interno, será deixada a cargo do leitor.

Antes do próximo resultado, uma definição.
Dado um subespaço \( E \) de um espaço com produto interno \( V \), definimos \[
    E^{ \perp } = \left\{ v \in V : \dotprod{v,w} = 0 \text{ para todo } w \in E \right\}
.\]

\begin{teo}[Projeção ortogonal]
    Se \( E \) é subespaço vetorial fechado de um espaço de Hilbert \( H \), então \[
        H = E \oplus E^{ \perp }
    .\]
    \label{teo: proj_ort}
\end{teo}
\begin{proof}
    Dado \( v \in H \), definimos \( \delta = \inf \left\{ \norm{ v - w } : w \in E \right\} \).
    Seja \( (w_{ n })_{ n\in\N } \) uma sequência de elementos de \( E \) tais que \( \norm{ v - w_{ n } } \to \delta \).
    Então, sendo \( k \) e \( \ell \) números naturais, aplicando a lei do paralelogramo para os vetores \( w_{ k } - v \) e \( w_{ \ell } - v \) obtemos: \[
        2 \norm{ w_{ k } - v }^2 + 2 \norm{ w_{ \ell } - v }^2
        = \norm{ w_{ k } + w_{ \ell } - 2v }^2 + \norm{ w_{ k } - w_{ \ell } }
    ,\]
    o que implica, remanejando e lembrando que \( (w_{ k } + w_{ \ell })/2 \in E \),
    \begin{align*}
        \norm{ w_{ k } - w_{ \ell } }
        &= 2 \norm{ w_{ k } - v }^2 + 2 \norm{ w_{ \ell } - v }^2 - 4 \norm{ (w_{ k } + w_{ \ell })/2 - v }^2 \\
        &\leq 2 \norm{ w_{ k } - v }^2 + 2 \norm{ w_{ \ell } - v }^2 - 4 \delta^2
    .\end{align*}
    Com isso, concluímos que \( (w_{ n }) \) é uma sequência de Cauchy.
    Como \( H \) é um espaço de Hilbert, temos que \( w_{ n } \to w \in E \), pois \( E \) é fechado e, pela continuidade da norma, temos \( \norm{ v - w } = \delta \).

    Intuitivamente, \( w \) é o elemento de \( E \) mais próximo de \( v \).
    Logo, é razoável esperar que ele seja a projeção ortogonal de \( v \) em \( E \).
    Para confirmar essa suspeita, devemos verificar que \( v - w \in E^{ \perp } \).
    De fato, para todo \( \lambda \in \K \) e todo \( u \in E \) temos \[
        \norm{ (v - w) + \lambda u }
        = \norm{ v + (-w + \lambda u) }
        \geq \delta
        = \norm{ v - w }
    .\]
    Portanto, pelo lema \ref{lem: perp_equiv}, concluímos que \( v - w \in E^{ \perp } \).
    Sendo assim, temos \( v = w + (v - w) \), onde \( w \in E \) e \( v - w \in E^{ \perp } \).
    Para mostrar a unicidade dessa decomposição, suponha que \( v = w_{ 1 } + u_{ 1 } = w_{ 2 } + u_{ 2 } \), onde \( w_{ 1 }, w_{ 2 } \in E \) e \( u_{ 1 }, u_{ 2 } \in E^{ \perp } \).
    Então \[
        w_{ 1 } - w_{ 2 } = u_{ 2 } - u_{ 1 } \in E \cap E^{ \perp }
    ,\]
    o que implica \( w_{ 1 } - w_{ 2 } = u_{ 2 } - u_{ 1 } = 0 \), ou seja, \( w_{ 1 } = w_{ 2 } \) e \( u_{ 1 } = u_{ 2 } \).
\end{proof}

Agora podemos prosseguir para o principal resultado dessa subseção.

\begin{teo}[Teorema da representação de Riesz em espaços de Hilbert]
    Dado um espaço de Hilbert H e seu dual \( H^{ * } \), a função
    \begin{align*}
        \gamma : &H \to H^{ * } \\
            &v \mapsto \gamma(v) = f_{ v }
    ,\end{align*}
    tal que \( f_{ v } = \dotprod{v,\cdot} \), é uma isometria antilinear e sobrejetiva em \( H^{ * } \).
\end{teo}
\begin{proof}
    Para mostrar que \( \gamma \) é uma isometria, provaremos que \( \norm{ f_{ v } } = \norm{ v } \) para todo \( v \in H \).
    De fato, se \( v = 0 \) isso é evidente.
    Fixado \( v \in H \backslash \left\{ 0 \right\} \), pela desigualdade de Cauchy-Schwarz temos \( \abs{ f_{ v }(w) } = \abs{ \dotprod{v,w} } \leq \norm{ v } \norm{ w } \).
    Logo, \( \norm{ f_{ v } } \leq \norm{ v } \).
    Por outro lado, temos \( \norm{ v }^2 = \dotprod{v,v} = f_{ v }(v) \leq \norm{ f_{ v } } \norm{ v } \), ou seja, \( \norm{ v } \leq \norm{ f_{ v } } \) e acabamos.

    Além disso claramente \( \gamma \) é antilinear, pois \[
        \gamma (\alpha v + w) = \dotprod{\alpha v + w, \cdot} = \bar{\alpha} \dotprod{v, \cdot} + \dotprod{w,\cdot} = \bar{\alpha} \gamma(v) + \gamma(w)
    .\]

    A parte menos óbvia, e que dá nome ao teorema, é a da sobrejetividade.
    Repare que, com essa propriedade, todo funcional \( f \in H^{ * } \) fica unicamente associado a um determinado \( v \in H \) tal que \( f = \gamma(v) \).
    Dizemos, então, que \( v \) \emph{representa} \( f \).
    Para demonstrá-la, utilizaremos o teorema \ref{teo: proj_ort}.

    Dado \( f \in H^{ * } \), se \( f = 0 \) então naturalmente \( f = \gamma(0) \).
    Se \( f \neq 0 \), repare que \( \ker(f) \), o núcleo de \( f \), é um subespaço próprio fechado de \( H \).
    Pelo teorema \ref{teo: proj_ort}, temos \[
        H = \ker(f) \oplus \ker(f)^{ \perp }
    ,\]
    de onde concluímos que existe \( w \in \ker(f)^{ \perp } \) satisfazendo \( \norm{ w } = 1 \).
    Agora uma observação simples, porém fundamental: para todos \( u, v \in H \) temos \( f(v)u - f(u)v \in \ker(f) \).
    Logo, para todo \( v \in H \) temos \[
        \dotprod{w,f(v)w - f(w)v} = 0
    ,\]
    o que implica, desenvolvendo, \[
        f(v) = \dotprod{\overline{f(w)}w,v}
    .\]
    Sendo assim, \( f = \gamma(\overline{f(w)}w) \).
\end{proof}